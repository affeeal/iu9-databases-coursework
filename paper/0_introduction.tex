%%%%%%%%%%%%%%%%%%%%%%%%%%%%%%%%%%%%%%%%%%%%%%%%%%%%%%%%%%%%%%%%%%%%%%%%%%%%%%%%%%%%%%%%%%%%%%%%%%%%%%%%%%%%%%%%%%%%%%%%
\anonsection{ВВЕДЕНИЕ}

Базы данных являются неотъемлемой составляющей многих современных программных продуктов. Существует множество
разновидностей баз данных, различающихся способами хранения и обработки информации. В последнее время большое
распространение получают NoSQL-решения, среди которых особенно выделяются графовые базы данных. Они хранят информацию в
графовой модели, а это означает, что выборка данных относительно их связей с другими данными является базовой операцией,
выполняющейся эффективнее, чем в, например, реляционных аналогах. Графовые базы данных успешно применяются при решении
различных задач: генерация рекомендаций реального времени, управление идентификацией и доступом, выявление мошенничества
и многое другое.

Сегодня существует множество графовых СУБД. Среди наиболее распространённых --- Neo4j, ArangoDB, Amazon Neptune, Azure
Cosmos DB и другие. Они различаются поддерживаемыми моделями данных, реализацией графовой модели, направленностью на
аналитическую обработку или обработку транзакций, возможностями масштабирования. Одной из таких графовых СУБД является
Dgraph --- проект с открытым исходным кодом, ориентированный на распределённое использование, масштабирование, нативно
поддерживающий технологию GraphQL и OLTP-направленный. СУБД Dgraph развивается, и становится актуальным вопрос оценки
её производительности для возможности сравнения с аналогичными графовыми СУБД. Важнейший показатель такой оценки ---
эффективность выполнения графовых запросов.

Таким образом, целью курсового проекта является оценка производительности СУБД Dgraph при выполнении графовых запросов.
Курсовой проект выполняется в рамках сравнения производительности графовых СУБД Neo4j, ArangoDB и Dgraph, поэтому
результаты оценки также должны быть сравнимы. Для этого, в частности, используются единые, определённые предварительно,
наборы данных и графовых запросов. Также результаты оценки должны быть легко воспроизводимы, для чего важно
автоматизировать основные этапы работы: загрузку исходных данных и выполнение графовых запросов. При этом необходимо
минимизировать дублирование в итоговом программном коде.
%%%%%%%%%%%%%%%%%%%%%%%%%%%%%%%%%%%%%%%%%%%%%%%%%%%%%%%%%%%%%%%%%%%%%%%%%%%%%%%%%%%%%%%%%%%%%%%%%%%%%%%%%%%%%%%%%%%%%%%%
