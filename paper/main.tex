% !TeX TXS-program:bibliography = txs:///biber
\documentclass[14pt, russian]{scrartcl}
\let\counterwithout\relax
\let\counterwithin\relax
%\usepackage{lmodern}
\usepackage{float}
\usepackage{xcolor}
\usepackage{extsizes}
\usepackage{subfig}
\usepackage[export]{adjustbox}
\usepackage{tocvsec2} % возможность менять учитываемую глубину разделов в оглавлении
\usepackage[subfigure]{tocloft}
\usepackage[newfloat]{minted}
\captionsetup[listing]{position=top}

\AtBeginEnvironment{figure}{\vspace{0.5cm}}
\AtBeginEnvironment{table}{\vspace{0.5cm}}
\AtBeginEnvironment{listing}{\vspace{0.5cm}}
\AtBeginEnvironment{algorithm}{\vspace{0.5cm}}
\AtBeginEnvironment{minted}{\vspace{-0.5cm}}

\usepackage{fancyvrb}
\usepackage{ulem,bm,mathrsfs,ifsym} %зачеркивания, особо жирный стиль и RSFS начертание
\usepackage{sectsty} % переопределение стилей подразделов
%%%%%%%%%%%%%%%%%%%%%%%

%%% Поля и разметка страницы %%%
\usepackage{pdflscape}                              % Для включения альбомных страниц
\usepackage{geometry}                               % Для последующего задания полей
\geometry{a4paper,tmargin=2cm,bmargin=2cm,lmargin=3cm,rmargin=1cm} % тоже самое, но лучше

%%% Математические пакеты %%%
\usepackage{amsthm,amsfonts,amsmath,amssymb,amscd}  % Математические дополнения от AMS
\usepackage{mathtools}                              % Добавляет окружение multlined
\usepackage[perpage]{footmisc}
%\usepackage{times}

%%%% Установки для размера шрифта 14 pt %%%%
%% Формирование переменных и констант для сравнения (один раз для всех подключаемых файлов)%%
%% должно располагаться до вызова пакета fontspec или polyglossia, потому что они сбивают его работу
%\newlength{\curtextsize}
%\newlength{\bigtextsize}
%\setlength{\bigtextsize}{13pt}
\KOMAoptions{fontsize=14pt}

\makeatletter
\def\showfontsize{\f@size{} point}
\makeatother

%\makeatletter
%\show\f@size                                       % неплохо для отслеживания, но вызывает стопорение процесса, если документ компилируется без команды  -interaction=nonstopmode 
%\setlength{\curtextsize}{\f@size pt}
%\makeatother

%шрифт times
\usepackage{tempora}
%\usepackage{pscyr}
%\setmainfont[Ligatures={TeX,Historic}]{Times New Roman}

   %%% Решение проблемы копирования текста в буфер кракозябрами
%    \input glyphtounicode.tex
%    \input glyphtounicode-cmr.tex %from pdfx package
%    \pdfgentounicode=1
    \usepackage{cmap}                               % Улучшенный поиск русских слов в полученном pdf-файле
    \usepackage[T1]{fontenc}                       % Поддержка русских букв
    \usepackage[utf8]{inputenc}                     % Кодировка utf8
    \usepackage[english, main=russian]{babel}            % Языки: русский, английский
%   \IfFileExists{pscyr.sty}{\usepackage{pscyr}}{}  % Красивые русские шрифты
%\renewcommand{\rmdefault}{ftm}
%%% Оформление абзацев %%%
\usepackage{indentfirst}                            % Красная строка
%\usepackage{eskdpz}

%%% Таблицы %%%
\usepackage{longtable}                              % Длинные таблицы
\usepackage{diagbox}
\usepackage{multirow,makecell,array}                % Улучшенное форматирование таблиц
\usepackage{booktabs}                               % Возможность оформления таблиц в классическом книжном стиле (при правильном использовании не противоречит ГОСТ)

%%% Общее форматирование
\usepackage{soulutf8}                               % Поддержка переносоустойчивых подчёркиваний и зачёркиваний
\usepackage{icomma}                                 % Запятая в десятичных дробях



%%% Изображения %%%
\usepackage{graphicx}                               % Подключаем пакет работы с графикой
\usepackage{wrapfig}

%%% Списки %%%
\usepackage{enumitem}

%%% Подписи %%%
\usepackage{caption}                                % Для управления подписями (рисунков и таблиц) % Может управлять номерами рисунков и таблиц с caption %Иногда может управлять заголовками в списках рисунков и таблиц
%% Использование:
%\begin{table}[h!]\ContinuedFloat - чтобы не переключать счетчик
%\captionsetup{labelformat=continued}% должен стоять до самого caption
%\caption{}
% либо ручками \caption*{Продолжение таблицы~\ref{...}.} :)

%%% Интервалы %%%
\addto\captionsrussian{%
  \renewcommand{\listingname}{Листинг}%
}
%%% Счётчики %%%
\usepackage[figure,table,section]{totalcount}               % Счётчик рисунков и таблиц
\DeclareTotalCounter{lstlisting}
\usepackage{totcount}                               % Пакет создания счётчиков на основе последнего номера подсчитываемого элемента (может требовать дважды компилировать документ)
\usepackage{totpages}                               % Счётчик страниц, совместимый с hyperref (ссылается на номер последней страницы). Желательно ставить последним пакетом в преамбуле

%%% Продвинутое управление групповыми ссылками (пока только формулами) %%%
%% Кодировки и шрифты %%%

%   \newfontfamily{\cyrillicfont}{Times New Roman}
%   \newfontfamily{\cyrillicfonttt}{CMU Typewriter Text}
	%\setmainfont{Times New Roman}
	%\newfontfamily\cyrillicfont{Times New Roman}
	%\setsansfont{Times New Roman}                    %% задаёт шрифт без засечек
%	\setmonofont{Liberation Mono}               %% задаёт моноширинный шрифт
%    \IfFileExists{pscyr.sty}{\renewcommand{\rmdefault}{ftm}}{}
%%% Интервалы %%%
%linespread-реализация ближе к реализации полуторного интервала в ворде.
%setspace реализация заточена под шрифты 10, 11, 12pt, под остальные кегли хуже, но всё же ближе к типографской классике. 
\linespread{1.3}                    % Полуторный интервал (ГОСТ Р 7.0.11-2011, 5.3.6)
%\renewcommand{\@biblabel}[1]{#1}

%%% Гиперссылки %%%
\usepackage{hyperref}

%%% Выравнивание и переносы %%%
\sloppy                             % Избавляемся от переполнений
\clubpenalty=10000                  % Запрещаем разрыв страницы после первой строки абзаца
\widowpenalty=10000                 % Запрещаем разрыв страницы после последней строки абзаца

\makeatletter % малые заглавные, small caps shape
\let\@@scshape=\scshape
\renewcommand{\scshape}{%
  \ifnum\strcmp{\f@series}{bx}=\z@
    \usefont{T1}{cmr}{bx}{sc}%
  \else
    \ifnum\strcmp{\f@shape}{it}=\z@
      \fontshape{scsl}\selectfont
    \else
      \@@scshape
    \fi
  \fi}
\makeatother

%%% Подписи %%%
%\captionsetup{%
%singlelinecheck=off,                % Многострочные подписи, например у таблиц
%skip=2pt,                           % Вертикальная отбивка между подписью и содержимым рисунка или таблицы определяется ключом
%justification=centering,            % Центрирование подписей, заданных командой \caption
%}
%%%        Подключение пакетов                 %%%
\usepackage{ifthen}                 % добавляет ifthenelse
%%% Инициализирование переменных, не трогать!  %%%
\newcounter{intvl}
\newcounter{otstup}
\newcounter{contnumeq}
\newcounter{contnumfig}
\newcounter{contnumtab}
\newcounter{pgnum}
\newcounter{bibliosel}
\newcounter{chapstyle}
\newcounter{headingdelim}
\newcounter{headingalign}
\newcounter{headingsize}
\newcounter{tabcap}
\newcounter{tablaba}
\newcounter{tabtita}
%%%%%%%%%%%%%%%%%%%%%%%%%%%%%%%%%%%%%%%%%%%%%%%%%%

%%% Область упрощённого управления оформлением %%%

%% Интервал между заголовками и между заголовком и текстом
% Заголовки отделяют от текста сверху и снизу тремя интервалами (ГОСТ Р 7.0.11-2011, 5.3.5)
\setcounter{intvl}{3}               % Коэффициент кратности к размеру шрифта

%% Отступы у заголовков в тексте
\setcounter{otstup}{0}              % 0 --- без отступа; 1 --- абзацный отступ

%% Нумерация формул, таблиц и рисунков
\setcounter{contnumeq}{1}           % Нумерация формул: 0 --- пораздельно (во введении подряд, без номера раздела); 1 --- сквозная нумерация по всей диссертации
\setcounter{contnumfig}{1}          % Нумерация рисунков: 0 --- пораздельно (во введении подряд, без номера раздела); 1 --- сквозная нумерация по всей диссертации
\setcounter{contnumtab}{1}          % Нумерация таблиц: 0 --- пораздельно (во введении подряд, без номера раздела); 1 --- сквозная нумерация по всей диссертации

%% Оглавление
\setcounter{pgnum}{0}               % 0 --- номера страниц никак не обозначены; 1 --- Стр. над номерами страниц (дважды компилировать после изменения)

%% Библиография
\setcounter{bibliosel}{1}           % 0 --- встроенная реализация с загрузкой файла через движок bibtex8; 1 --- реализация пакетом biblatex через движок biber

%% Текст и форматирование заголовков
\setcounter{chapstyle}{1}           % 0 --- разделы только под номером; 1 --- разделы с названием "Глава" перед номером
\setcounter{headingdelim}{1}        % 0 --- номер отделен пропуском в 1em или \quad; 1 --- номера разделов и приложений отделены точкой с пробелом, подразделы пропуском без точки; 2 --- номера разделов, подразделов и приложений отделены точкой с пробелом.

%% Выравнивание заголовков в тексте
\setcounter{headingalign}{0}        % 0 --- по центру; 1 --- по левому краю

%% Размеры заголовков в тексте
\setcounter{headingsize}{0}         % 0 --- по ГОСТ, все всегда 14 пт; 1 --- пропорционально изменяющийся размер в зависимости от базового шрифта

%% Подпись таблиц
\setcounter{tabcap}{0}              % 0 --- по ГОСТ, номер таблицы и название разделены тире, выровнены по левому краю, при необходимости на нескольких строках; 1 --- подпись таблицы не по ГОСТ, на двух и более строках, дальнейшие настройки: 
%Выравнивание первой строки, с подписью и номером
\setcounter{tablaba}{2}             % 0 --- по левому краю; 1 --- по центру; 2 --- по правому краю
%Выравнивание строк с самим названием таблицы
\setcounter{tabtita}{1}             % 0 --- по левому краю; 1 --- по центру; 2 --- по правому краю

%%% Рисунки %%%
\DeclareCaptionLabelSeparator*{emdash}{~--- }             % (ГОСТ 2.105, 4.3.1)
\captionsetup[figure]{labelsep=emdash,font=onehalfspacing,position=bottom}

%%% Таблицы %%%
\ifthenelse{\equal{\thetabcap}{0}}{%
    \newcommand{\tabcapalign}{\raggedright}  % по левому краю страницы или аналога parbox
}

\ifthenelse{\equal{\thetablaba}{0} \AND \equal{\thetabcap}{1}}{%
    \newcommand{\tabcapalign}{\raggedright}  % по левому краю страницы или аналога parbox
}

\ifthenelse{\equal{\thetablaba}{1} \AND \equal{\thetabcap}{1}}{%
    \newcommand{\tabcapalign}{\centering}    % по центру страницы или аналога parbox
}

\ifthenelse{\equal{\thetablaba}{2} \AND \equal{\thetabcap}{1}}{%
    \newcommand{\tabcapalign}{\raggedleft}   % по правому краю страницы или аналога parbox
}

\ifthenelse{\equal{\thetabtita}{0} \AND \equal{\thetabcap}{1}}{%
    \newcommand{\tabtitalign}{\raggedright}  % по левому краю страницы или аналога parbox
}

\ifthenelse{\equal{\thetabtita}{1} \AND \equal{\thetabcap}{1}}{%
    \newcommand{\tabtitalign}{\centering}    % по центру страницы или аналога parbox
}

\ifthenelse{\equal{\thetabtita}{2} \AND \equal{\thetabcap}{1}}{%
    \newcommand{\tabtitalign}{\raggedleft}   % по правому краю страницы или аналога parbox
}

\DeclareCaptionFormat{tablenocaption}{\tabcapalign #1\strut}        % Наименование таблицы отсутствует
\ifthenelse{\equal{\thetabcap}{0}}{%
    \DeclareCaptionFormat{tablecaption}{\tabcapalign #1#2#3}
    \captionsetup[table]{labelsep=emdash}                       % тире как разделитель идентификатора с номером от наименования
}{%
    \DeclareCaptionFormat{tablecaption}{\tabcapalign #1#2\par%  % Идентификатор таблицы на отдельной строке
        \tabtitalign{#3}}                                       % Наименование таблицы строкой ниже
    \captionsetup[table]{labelsep=space}                        % пробельный разделитель идентификатора с номером от наименования
}
\captionsetup[table]{format=tablecaption,singlelinecheck=off,font=onehalfspacing,position=top,skip=-5pt}  % многострочные наименования и прочее
\DeclareCaptionLabelFormat{continued}{Продолжение таблицы~#2}
\setlength{\belowcaptionskip}{.2cm}
\setlength{\intextsep}{0ex}

%%% Подписи подрисунков %%%
\renewcommand{\thesubfigure}{\asbuk{subfigure}}           % Буквенные номера подрисунков
\captionsetup[subfigure]{font={normalsize},               % Шрифт подписи названий подрисунков (не отличается от основного)
    labelformat=brace,                                    % Формат обозначения подрисунка
    justification=centering,                              % Выключка подписей (форматирование), один из вариантов            
}
%\DeclareCaptionFont{font12pt}{\fontsize{12pt}{13pt}\selectfont} % объявляем шрифт 12pt для использования в подписях, тут же надо интерлиньяж объявлять, если не наследуется
%\captionsetup[subfigure]{font={font12pt}}                 % Шрифт подписи названий подрисунков (всегда 12pt)

%%% Настройки гиперссылок %%%

\definecolor{linkcolor}{rgb}{0.0,0,0}
\definecolor{citecolor}{rgb}{0,0.0,0}
\definecolor{urlcolor}{rgb}{0,0,0}

\hypersetup{
    linktocpage=true,           % ссылки с номера страницы в оглавлении, списке таблиц и списке рисунков
%    linktoc=all,                % both the section and page part are links
%    pdfpagelabels=false,        % set PDF page labels (true|false)
    plainpages=true,           % Forces page anchors to be named by the Arabic form  of the page number, rather than the formatted form
    colorlinks,                 % ссылки отображаются раскрашенным текстом, а не раскрашенным прямоугольником, вокруг текста
    linkcolor={linkcolor},      % цвет ссылок типа ref, eqref и подобных
    citecolor={citecolor},      % цвет ссылок-цитат
    urlcolor={urlcolor},        % цвет гиперссылок
    pdflang={ru},
}
\urlstyle{same}
%%% Шаблон %%%
%\DeclareRobustCommand{\todo}{\textcolor{red}}       % решаем проблему превращения названия цвета в результате \MakeUppercase, http://tex.stackexchange.com/a/187930/79756 , \DeclareRobustCommand protects \todo from expanding inside \MakeUppercase
\setlength{\parindent}{2.5em}                       % Абзацный отступ. Должен быть одинаковым по всему тексту и равен пяти знакам (ГОСТ Р 7.0.11-2011, 5.3.7).

%%% Списки %%%
% Используем дефис для ненумерованных списков (ГОСТ 2.105-95, 4.1.7)
%\renewcommand{\labelitemi}{\normalfont\bfseries~{---}} 
\renewcommand{\labelitemi}{\bfseries~{---}} 
\setlist{nosep,%                                    % Единый стиль для всех списков (пакет enumitem), без дополнительных интервалов.
    labelindent=\parindent,leftmargin=*%            % Каждый пункт, подпункт и перечисление записывают с абзацного отступа (ГОСТ 2.105-95, 4.1.8)
}
%%%%%%%%%%%%%%%%%%%%%%
%\usepackage{xltxtra} % load xunicode

\usepackage{ragged2e}
\usepackage[explicit]{titlesec}
\usepackage{placeins}
\usepackage{xparse}
\usepackage{csquotes}

\usepackage{listingsutf8}
\usepackage{url} %пакеты расширений
\usepackage{algorithm, algorithmicx}
\usepackage[noend]{algpseudocode}
\usepackage{blkarray}
\usepackage{chngcntr}
\usepackage{tabularx}
\usepackage[backend=biber, 
    bibstyle=gost-numeric,
    citestyle=nature]{biblatex}
\newcommand*\template[1]{\text{<}#1\text{>}}
\addbibresource{biblio.bib}
  
\titleformat{name=\section,numberless}[block]{\normalfont\Large\centering}{}{0em}{#1}
\titleformat{\section}[block]{\normalfont\Large\bfseries\raggedright}{}{0em}{\thesection\hspace{0.25em}#1}
\titleformat{\subsection}[block]{\normalfont\Large\bfseries\raggedright}{}{0em}{\thesubsection\hspace{0.25em}#1}
\titleformat{\subsubsection}[block]{\normalfont\large\bfseries\raggedright}{}{0em}{\thesubsubsection\hspace{0.25em}#1}

\let\Algorithm\algorithm
\renewcommand\algorithm[1][]{\Algorithm[#1]\setstretch{1.5}}
%\renewcommand{\listingscaption}{Листинг}

\usepackage{pifont}
\usepackage{calc}
\usepackage{suffix}
\usepackage{csquotes}
\DeclareQuoteStyle{russian}
    {\guillemotleft}{\guillemotright}[0.025em]
    {\quotedblbase}{\textquotedblleft}
\ExecuteQuoteOptions{style=russian}
\newcommand{\enq}[1]{\enquote{#1}}  
\newcommand{\eng}[1]{\begin{english}#1\end{english}}
% Подчиненные счетчики в окружениях http://old.kpfu.ru/journals/izv_vuz/arch/sample1251.tex
\newcounter{cTheorem} 
\newcounter{cDefinition}
\newcounter{cConsequent}
\newcounter{cExample}
\newcounter{cLemma}
\newcounter{cConjecture}
\newtheorem{Theorem}{Теорема}[cTheorem]
\newtheorem{Definition}{Определение}[cDefinition]
\newtheorem{Consequent}{Следствие}[cConsequent]
\newtheorem{Example}{Пример}[cExample]
\newtheorem{Lemma}{Лемма}[cLemma]
\newtheorem{Conjecture}{Гипотеза}[cConjecture]

\renewcommand{\theTheorem}{\arabic{Theorem}}
\renewcommand{\theDefinition}{\arabic{Definition}}
\renewcommand{\theConsequent}{\arabic{Consequent}}
\renewcommand{\theExample}{\arabic{Example}}
\renewcommand{\theLemma}{\arabic{Lemma}}
\renewcommand{\theConjecture}{\arabic{Conjecture}}
%\makeatletter
\NewDocumentCommand{\Newline}{}{\text{\\}}
\newcommand{\sequence}[2]{\ensuremath \left(#1,\ \dots,\ #2\right)}

\definecolor{mygreen}{rgb}{0,0.6,0}
\definecolor{mygray}{rgb}{0.5,0.5,0.5}
\definecolor{mymauve}{rgb}{0.58,0,0.82}
\renewcommand{\listalgorithmname}{Список алгоритмов}
\floatname{algorithm}{Листинг}
\renewcommand{\lstlistingname}{Листинг}
\renewcommand{\thealgorithm}{\arabic{algorithm}}

\newcommand{\refAlgo}[1]{(листинг \ref{#1})}
\newcommand{\refImage}[1]{(рисунок \ref{#1})}

\renewcommand{\theenumi}{\arabic{enumi}.}% Меняем везде перечисления на цифра.цифра	
\renewcommand{\labelenumi}{\arabic{enumi}.}% Меняем везде перечисления на цифра.цифра
\renewcommand{\theenumii}{\arabic{enumii}}% Меняем везде перечисления на цифра.цифра
\renewcommand{\labelenumii}{(\arabic{enumii})}% Меняем везде перечисления на цифра.цифра
\renewcommand{\theenumiii}{\roman{enumiii}}% Меняем везде перечисления на цифра.цифра
\renewcommand{\labelenumiii}{(\roman{enumiii})}% Меняем везде перечисления на цифра.цифра
%\newfontfamily\AnkaCoder[Path=src/fonts/]{AnkaCoder-r.ttf}
\renewcommand{\labelitemi}{---}
\renewcommand{\labelitemii}{---}

%\usepackage{courier}

\lstdefinelanguage{Refal}{
  alsodigit = {.,<,>},
  morekeywords = [1]{$ENTRY},
  morekeywords = [2]{Go, Put, Get, Open, Close, Arg, Add, Sub, Mul, Div, Symb, Explode, Implode},
  %keyword4
  morekeywords = [3]{<,>},
  %keyword5
  morekeywords = [4]{e.,t.,s.},
  sensitive = true,
  morecomment = [l]{*},
  morecomment = [s]{/*}{*/},
  commentstyle = \color{mygreen},
  morestring = [b]",
  morestring = [b]',
  stringstyle = \color{purple}
}

\makeatletter
\def\p@subsection{}
\def\p@subsubsection{\thesection\,\thesubsection\,}
\makeatother
\newcommand{\prog}[1]{{\ttfamily\small#1}}
\lstset{ %
  backgroundcolor=\color{white},   % choose the background color; you must add \usepackage{color} or \usepackage{xcolor}
  basicstyle=\ttfamily\footnotesize, 
  %basicstyle=\footnotesize\AnkaCoder,        % the size of the fonts that are used for the code
  breakatwhitespace=false,         % sets if automatic breaks shoulbd only happen at whitespace
  breaklines=true,                 % sets automatic line breaking
  captionpos=top,                    % sets the caption-position to bottom
  commentstyle=\color{mygreen},    % comment style
  deletekeywords={...},            % if you want to delete keywords from the given language
  escapeinside={\%*}{*)},          % if you want to add LaTeX within your code
  extendedchars=true,              % lets you use non-ASCII characters; for 8-bits encodings only, does not work with UTF-8
  inputencoding=utf8,
  frame=single,                    % adds a frame around the code
  keepspaces=true,                 % keeps spaces in text, useful for keeping indentation of code (possibly needs columns=flexible)
  keywordstyle=\bf,       % keyword style
  language=Refal,                    % the language of the code
  morekeywords={<,>,$ENTRY,Go,Arg, Open, Close, e., s., t., Get, Put}, 
  							       % if you want to add more keywords to the set
  numbers=left,                    % where to put the line-numbers; possible values are (none, left, right)
  numbersep=5pt,                   % how far the line-numbers are from the code
  xleftmargin=25pt,
  xrightmargin=25pt,
  numberstyle=\small\color{black}, % the style that is used for the line-numbers
  rulecolor=\color{black},         % if not set, the frame-color may be changed on line-breaks within not-black text (e.g. comments (green here))
  showspaces=false,                % show spaces everywhere adding particular underscores; it overrides 'showstringspaces'
  showstringspaces=false,          % underline spaces within strings only
  showtabs=false,                  % show tabs within strings adding particular underscores
  stepnumber=1,                    % the step between two line-numbers. If it's 1, each line will be numbered
  stringstyle=\color{mymauve},     % string literal style
  tabsize=8,                       % sets default tabsize to 8 spaces
  title=\lstname                   % show the filename of files included with \lstinputlisting; also try caption instead of title
}
\newcommand{\anonsection}[1]{\cleardoublepage
\phantomsection
\addcontentsline{toc}{section}{\protect\numberline{}#1}
\section*{#1}\vspace*{2.5ex} % По госту положены 3 пустые строки после заголовка ненумеруемого раздела
}
\newcommand{\sectionbreak}{\clearpage}
\renewcommand{\sectionfont}{\normalsize} % Сбиваем стиль оглавления в стандартный
\renewcommand{\cftsecleader}{\cftdotfill{\cftdotsep}} % Точки в оглавлении напротив разделов

\renewcommand{\cftsecfont}{\normalfont\large} % Переключение на times в содержании
\renewcommand{\cftsubsecfont}{\normalfont\large} % Переключение на times в содержании

\setlength{\cftsecindent}{0pt}% Убираем отступы в содержании для \section
\setlength{\cftsubsecindent}{0pt}% Убираем отступы в содержании для \subsection
\setlength{\cftsubsubsecindent}{0pt}% Убираем отступы в содержании для \subsubsection

\usepackage{caption} 
%\captionsetup[table]{justification=raggedleft} 
%\captionsetup[figure]{justification=centering,labelsep=endash}
\usepackage{amsmath}    % \bar    (матрицы и проч. ...)
\usepackage{amsfonts}   % \mathbb (символ для множества действительных чисел и проч. ...)
\usepackage{mathtools}  % \abs, \norm
    \DeclarePairedDelimiter\abs{\lvert}{\rvert} % операция модуля
    \DeclarePairedDelimiter\norm{\lVert}{\rVert} % операция нормы
\DeclareTextCommandDefault{\textvisiblespace}{%
  \mbox{\kern.06em\vrule \@height.3ex}%
  \vbox{\hrule \@width.3em}%
  \hbox{\vrule \@height.3ex}}    
\newsavebox{\spacebox}
\begin{lrbox}{\spacebox}
\verb*! !
\end{lrbox}
\newcommand{\aspace}{\usebox{\spacebox}}
\DeclareTotalCounter{listing}

\makeatletter
\renewcommand*{\p@subsubsection}{}
\makeatother
    
\begin{document}
\sloppy

\def\figurename{Рисунок}

\begin{titlepage}
\thispagestyle{empty}
\newpage

\vspace*{-30pt}
\hspace{-45pt}
\begin{minipage}{0.17\textwidth}
\hspace*{-20pt}\centering
\includegraphics[width=1.3\textwidth]{emblem.png}
\end{minipage}
\begin{minipage}{0.82\textwidth}\small \textbf{
\vspace*{-0.7ex}
\hspace*{-10pt}\centerline{Министерство науки и высшего образования Российской Федерации}
\vspace*{-0.7ex}
\centerline{Федеральное государственное бюджетное образовательное учреждение }
\vspace*{-0.7ex}
\centerline{высшего образования}
\vspace*{-0.7ex}
\centerline{<<Московский государственный технический университет}
\vspace*{-0.7ex}
\centerline{имени Н.Э. Баумана}
\vspace*{-0.7ex}
\centerline{(национальный исследовательский университет)>>}
\vspace*{-0.7ex}
\centerline{(МГТУ им. Н.Э. Баумана)}}
\end{minipage}

\vspace{-2pt}
\hspace{-34.5pt}\rule{\textwidth}{2.5pt}

\vspace*{-20.3pt}
\hspace{-34.5pt}\rule{\textwidth}{0.4pt}
 
\vspace{0.5ex}
\noindent \small ФАКУЛЬТЕТ\hspace{80pt} <<Информатика и системы управления>>

\vspace*{-16pt}
\hspace{35pt}\rule{0.855\textwidth}{0.4pt}

\vspace{0.5ex}
\noindent \small КАФЕДРА\hspace{50pt} <<Теоретическая информатика и компьютерные технологии>>

\vspace*{-16pt}
\hspace{25pt}\rule{0.875\textwidth}{0.4pt}
 
 
\vspace{3em}
 
\begin{center}
\Large \bf{РАСЧЕТНО-ПОЯСНИТЕЛЬНАЯ ЗАПИСКА\\\textbf{\textit{К КУРСОВОЙ РАБОТЕ\\НА ТЕМУ:}} \\}
\end{center}

\vspace*{-6ex} 
\begin{center}
\Large{\textit{\textbf{<<Оценка производительности СУБД Dgraph}}}

\vspace*{-3ex}
\rule{0.9\textwidth}{1.2pt}

\Large{\textit{\textbf{при выполнении графовых запросов>>}}}
\vspace*{-3ex}
\vspace*{-0.2ex}
\rule{0.9\textwidth}{1.2pt}

\vspace*{-0.2ex}
\rule{0.9\textwidth}{1.2pt}

\vspace*{-0.2ex}
\rule{0.9\textwidth}{1.2pt}

\vspace*{-0.2ex}
\rule{0.9\textwidth}{1.2pt}
\end{center}
 
\vspace{\fill}
 

\newlength{\ML}
\settowidth{\ML}{«\underline{\hspace{0.7cm}}» \underline{\hspace{2cm}}}

\noindent Студент \underline{\hspace{1.5cm}} \hfill \underline{\hspace{4cm}}\quad
\underline{\hspace{4cm}}

\vspace{-2.1ex}
\noindent\hspace{9ex}\scriptsize{(Группа)}\normalsize\hspace{170pt}\hspace{2ex}\scriptsize{(Подпись, дата)}\normalsize\hspace{30pt}\hspace{6ex}\scriptsize{(И.О. Фамилия)}\normalsize

\bigskip

\noindent Руководитель  \hfill \underline{\hspace{4cm}}\quad
\underline{\hspace{4cm}}

\vspace{-2ex}
\noindent\hspace{13.5ex}\normalsize\hspace{170pt}\hspace{2ex}\scriptsize{(Подпись, дата)}\normalsize\hspace{30pt}\hspace{6ex}\scriptsize{(И.О. Фамилия)}\normalsize
\vfill

%\vspace{\fill}
 


\begin{center}
\textsl{2024 г.}
\end{center}
\end{titlepage}

%\renewcommand{\ttdefault}{pcr}

\setlength{\tabcolsep}{3pt}
\newpage
\setcounter{page}{2}

\renewcommand\contentsname{\hfill{\normalfont{СОДЕРЖАНИЕ}}\hfill}
\tableofcontents

%%%%%%%%%%%%%%%%%%%%%%%%%%%%%%%%%%%%%%%%%%%% TEXT WIDTH %%%%%%%%%%%%%%%%%%%%%%%%%%%%%%%%%%%%%%%%%%%%
\anonsection{ВВЕДЕНИЕ}

Базы данных являются неотъемлемой составляющей многих современных программных продуктов. Существует
множество разновидностей баз данных, различающихся способами хранения и обработки информации. В
последнее время большое распространение получают NoSQL-решения, среди которых особенно выделяются
графовые базы данных. Они хранят информацию в графовой модели, а это означает, что выборка данных
относительно их связей с другими данными является базовой операцией, выполняющейся эффективнее, чем в,
например, реляционных аналогах. Графовые базы данных успешно применяются при решении различных задач:
генерация рекомендаций реального времени, управление идентификацией и доступом, выявление мошенничества
и многое другое.

Сегодня существует множество графовых СУБД. Среди наиболее распространённых --- Neo4j, ArangoDB,
Amazon Neptune, Azure Cosmos DB и другие. Они различаются поддерживаемыми моделями данных, реализацией
графовой модели, направленностью на аналитическую обработку или обработку транзакций, возможностями
масштабирования. Одной из таких графовых СУБД является Dgraph --- проект с открытым исходным кодом,
ориентированный на распределённое использование и масштабирование, нативно поддерживающий технологию
GraphQL и OLTP-направленный. СУБД Dgraph развивается, и становится актуальным вопрос оценки
её производительности для возможности сравнения с аналогичными графовыми СУБД. Важнейший показатель
такой оценки --- эффективность выполнения графовых запросов.

Таким образом, целью курсового проекта является оценка производительности СУБД Dgraph при
выполнении графовых запросов. Курсовой проект выполняется в рамках сравнения производительности
графовых СУБД Neo4j, ArangoDB и Dgraph, поэтому результаты оценивания также должны быть сравнимы.
Для этого, в частности, будут использоваться единые, определённые предварительно, наборы данных и
графовых запросов. Также результаты оценивания должны быть легко воспроизводимы, для чего важно
автоматизировать основные этапы работы: загрузку исходных данных и выполнение графовых запросов.
При этом необходимо минимизировать дублирование в итоговом программном коде.

%%%%%%%%%%%%%%%%%%%%%%%%%%%%%%%%%%%%%%%%%%%% TEXT WIDTH %%%%%%%%%%%%%%%%%%%%%%%%%%%%%%%%%%%%%%%%%%%%

%%%%%%%%%%%%%%%%%%%%%%%%%%%%%%%%%%%%%%%%%%%% TEXT WIDTH %%%%%%%%%%%%%%%%%%%%%%%%%%%%%%%%%%%%%%%%%%%%
\section {Обзор СУБД Dgraph}

\subsection{Графовая модель данных}

Dgraph~\cite{dgraph} является \textit{нативной} графовой СУБД, т.е. графовая модель данных в Dgraph
--- единственная\footnote{Существуют также мультимодельные СУБД. Например, в ArangoDB помимо графовой
поддерживаются документо-ориентированная модель и модель <<ключ-значение>>.}. Современные графовые
СУБД поддерживают, как правило, одну из двух следующих реализаций графовой модели:
\begin{itemize}
  \item \textit{граф свойств} (property graph), хранящий вершины (сущности), рёбра (связи между
    сущностями) и свойства, описывающие вершины или рёбра. Именно эта модель лежит в основе Dgraph;
  \item \textit{RDF-граф}, использующий модель субъект-предикат-объект для хранения информации.
    Dgraph поддерживает формат RDF для импорта и экспорта данных.
\end{itemize}

Связи между сущностями в Dgraph направленные, что позволяет оптимально совершать обход от одной
сущности к другой. Для двунаправленной связи нужно, что естественно, установить связь между
сущностями в обе стороны.

\subsection{Архитектура}

Dgraph эффективно масштабируется для работы с большими объёмами данных, поскольку проектируется с
самого начала для распределённого использования. В основе его работы --- кластер взаимодействующих
друг с другом серверных узлов, образующих единое логическое хранилище данных. В Dgraph выделяются
два типа серверных узлов:
\begin{itemize}
  \item \textit{Zero-узлы} содержат метаданные кластера Dgraph, координируют распределённые
    транзакции и распределяют данные между группами серверов;
  \item \textit{Alpha-узлы} хранят данные графа и индексы. Важно отметить, что Alpha-узлы хранят
    и индексируют \textit{предикаты}, представляющие связи между данными. Такой подход позволяет
    Dgraph выполнять запросы глубины $N$ к базе данных ровно за $N$ сетевых переходов.
\end{itemize}

Каждый кластер Dgraph должен иметь как минимум один Zero- и один Alpha-узел.

\subsection{Поддержка GraphQL}

В Dgraph встроена поддержка технологии GraphQL~\cite{graphql}. GraphQL --- язык запросов и
манипулирования данными для API, а также серверная среда выполнения запросов с открытым исходным
кодом. GraphQL позволяет клиенту в декларативной форме описывать точные данные, требуемые от API.
Вместо нескольких конечных точек (endpoints), возвращающих раздельные данные, сервер GraphQL
предоставляет единственнную, возвращающую ровно ту информацию, которую запрашивает клиент. Сервер
GraphQL может извлекать данные из различных источников, представляя результат в виде единого графа,
то есть GraphQL не привязан к какой-либо базе данных или иному механизму хранения информации.

Для запуска сервиса GraphQL необходимо определить \textit{схему} --- типы данных и их поля, и для
каждого такого поля --- его \textit{разрешающую функцию} (resolver), вычисляющую значение поля.
Принимая запрос, GraphQL его валидидирует соответственно определённой схеме, и в случае успеха
выполняет, вызывая соответствующие полям запроса разрешающие функции. Сервер возвращает данные, в
точности отвечающие исходному запросу.

Нативная поддержка GraphQL выделяет Dgraph на фоне его аналогов. Пользователю достаточно описать
схему GraphQL, и GraphQL API будет готов к работе. Dgraph автоматически определит разрешающие
функции: в основе их реализации --- простое следование по связям в графе от вершины к вершине и от
вершины к полю, что и обеспечивает нативную графовую производительность.

\subsection{Поддержка DQL}

На практике часто возникает необходимость в выполнении сложных запросов, не поддерживаемых
спецификацией GraphQL. Для этого разработан Dgraph Query Language (DQL) --- проприетарный язык
запросов Dgraph, синтаксически напоминающий GraphQL, но обладающий большей выразительной силой.
Кажется, что при наличии DQL нет необходимости в поддержке GraphQL. На деле же языки различаются
своим назначением~\cite{graphqlVsDql} и оба могут обоснованно использоваться в рамках одного
проекта.

Сервис GraphQL предоставляет API для контролируемого и защищённого доступа внешних клиентов к базе
данных. Dgraph поддерживает набор директив для применения в схеме GraphQL, посредством которых
выражаются правила авторизации доступа к данным и иные ограничения бизнес-логики.

DQL следует использовать как язык <<сырых>> запросов к базе данных, подобных, например, языку SQL. 
Предполагается, что Dgraph доверяет клиентам DQL. DQL позволяет выполнять продвинутые запросы к
базе даных и лучше подходит для работы с большими данными, допускающими пакетную обработку.

\subsection{Схема DQL}

Схема DQL содержит данные о типах \textit{предикатов}. Предикат --- это наименьший фрагмент
информации о сущности. Предикат может содержать литеральное значение или связь с другой сущностью.
Рассмотрим на примере:
\begin{itemize}
  \item пусть мы сохраняем имя сущности --- ''Петя''. Тогда мы можем использовать некоторый
    предикат \texttt{name}, значением которого будет строка ''Петя'';
  \item пусть мы сохраняем отношение сущностей: Петя знает Машу. Тогда мы можем использовать некоторый
    предикат \texttt{knows}, значением которого будет идентификатор сущности, представляющей Машу.
\end{itemize}

Объявление предиката в схеме имеет вид
\begin{Verbatim}
имя_предиката: тип_предиката директивы .
\end{Verbatim}
где \texttt{директивы} опциональны. Некоторые из поддерживаемых типов предикатов:
\begin{itemize}
    \item \texttt{int} --- целочисленный (32 бита);
    \item \texttt{float} --- вещественнозначный (64 бита);
    \item \texttt{string} --- строковый;
    \item \texttt{uid} --- численный универсальный идентификатор.
\end{itemize}
Предикат также может содержать список элементов --- для этого тип в объявлении обрамляется
квадратными скобками.

Посредством директив, в частности, определяются индексы на предикаты (директива \texttt{@index})
или обратные связи для \texttt{uid} (директива \texttt{@reversed}).

\subsection{DQL и графовые алгоритмы}

Хотя язык DQL подобен языку SQL по своей направленности, DQL ограничен в своей выразительности и
не пригоден для реализации графовых алгоритмов. Например, запрос с подсчётом всех треугольников в
графе не выразим на языке DQL.

DQL поддерживает альтернативу классическому обходу графа в ширину --- \textit{рекурсивный запрос}.
В запросе указываются глубина и предикаты схемы DQL, по которым совершается обход. В запросе
(не только рекурсивном) может быть указана директива \texttt{@ignorereflex}: она принудительно
удаляет дочерние узлы, которые достижимы из самих себя как родительские.

DQL также поддерживает поиск кратчайшего пути между двумя вершинами. В теле запроса указываются
предикаты, по которым совершается обход.

\subsection{Формат RDF}

В Dgraph встроена поддержка формата RDF при создании, импорте и экспорте данных. Resource Description
Framework (RDF)~\cite{rdf} --- семантический веб-стандарт обмена данными, посредством которого
выражаются утверждения о ресурсах. Утверждения описываются в форме троек, оканчивающихся точкой:
\begin{Verbatim}
<субъект> <предикат> <объект> .
\end{Verbatim}

Каждая тройка представляет некоторый факт о сущности. В Dgraph \texttt{<субъект>} всегда является
сущностью и имеет тип \texttt{uid}, \texttt{<объект>} может быть или другой сущностью, или литеральным
значением, а \texttt{<предикат>} определяет отношение между субъектом и объектом. Например,
\begin{Verbatim}
<0x01> <name> "Петя" .
<0x01> <knows> <0x02> .
\end{Verbatim}

При создании сущности могут быть ассоцированы с некоторыми временными именами, что позволит Dgraph
самостоятельно назначить им UID. Для этого имена предваряются строчкой <<\_:>>. Таким образом,
все упоминания одного и того же имени, например, \texttt{\_:Petya}, будут соответствовать одной и
той же сущности в рамках операции. Идентификаторы такого формата называются \textit{blank}-идентификаторами.

Dgraph работает с собственным расширением формата RDF, позволяющим назначать свойства не только
сущностям (субъектам), но и отношениям (предикатам). Свойства отношений или \textit{фасеты} (facets) описываются парами
ключ-значение, перечисляемыми в круглых скобках после указания объекта RDF. Например,
\begin{Verbatim}
_:Petya <car> "A123BC" (since=2020-01-01T12:00:00, first=true) .
\end{Verbatim}

Ключи фасетов должны быть уникальными. Все фасеты должны указываться в рамках одной тройки RDF
(иначе произойдёт перезапись). Допустимы следующие типы фасетов: \texttt{string}, \texttt{bool},
\texttt{int} (32 бита), \texttt{float} (64 бита) и \texttt{dateTime}. Типы фасетов не указываются
явно, Dgraph автоматически определяет их по формату значения.

\subsection{Формат JSON}

При вставке, обновлении данных или возврате структур также может использоваться формат JSON.
Вложенность в JSON-объектах выражает отношения между сущностями. Например,
\begin{Verbatim}
{"name": "Петя", "homeAddress": {"street": "ул. Пушкина"}}
\end{Verbatim}
интуитивно соответствует сущностям некоторых типов \texttt{Person}, \texttt{Address} и отношению
\texttt{homeAddress} между ними.

Данные в форматах RDF и JSON приводятся к единому внутреннему представлению Dgraph, поэтому их
использование одинаково эффективно. Однако RDF, в отличие от JSON, позволяет явно назначать объектам
типы данных, если это необходимо.

\subsection{Загрузка данных}

Для загрузки больших объёмов данных целесообразно использовать специализированные для импорта
инструменты Dgraph. Таких инструментов два: Live Loader и Bulk Loader.

Live Loader используется для загрузки данных в работающем экземпляре Dgraph, где уже может содержаться
некоторая информация. Исходные данные преобразуются загрузчиком в \textit{мутации} (mutations)
Dgraph --- запросы на создание или изменение данных --- для дальнейшей отправки кластеру. Live Loader
позволяет управлять назначением UID добавляемым сущностям, а также обновлять существующие данные.
Перед использованием Live Loader необходимо определить схему Dgraph, чтобы создать индексы на предикаты
и уменьшить общее время загрузки.

Bulk Loader используется только для загрузки данных в новый кластер и не может быть запущен в
работающем экземпляре Dgraph. Bulk Loader работает значительно быстрее Live Loader, и поэтому лучше
подходит для импорта больших наборов данных в новый кластер. При работе с Bulk Loader должны быть
запущены только Zero-узлы Dgraph; Alpha-узлы запускаются по завершении загрузки данных. При
использовании Bulk Loader схема Dgraph может быть передана загрузчику вместе с данными.

И Live Loader, и Bulk Loader принимают данные в формате JSON или в Dgraph-расширении формата RDF.

%%%%%%%%%%%%%%%%%%%%%%%%%%%%%%%%%%%%%%%%%%%% TEXT WIDTH %%%%%%%%%%%%%%%%%%%%%%%%%%%%%%%%%%%%%%%%%%%%

%%%%%%%%%%%%%%%%%%%%%%%%%%%%%%%%%%%%%%%%%%%%%%%%%%%%%%%%%%%%%%%%%%%%%%%%%%%%%%%%%%%%%%%%%%%%%%%%%%%%%%%%%%%%%%%%%%%%%%%%
\section{Формирование запросов и тестовых данных} \label{forming}

Для оценки производительности СУБД Dgraph над тестовыми данными выполняются следующие типы графовых
запросов\footnote{Запросы обозначаются строчными буквами латинского алфавита для удобной ссылки на них в дальнейшем.}:
\begin{enumerate}[label=(\alph*)]
  \item рекурсивный запрос с фильтрацией по вершинам на пути; \label{query1}
  \item выбор всех вершин с заданным значением поля; \label{query2}
  \item выбор всех вершин с заданным значением поля с фильтрацией по связям и степени вершины; \label{query3}
  \item подсчёт для вершин агрегации некоторого параметра по соседним вершинам с ограничением на значение параметра;
    \label{query4}
  \item поиск кратчайшего пути между вершинами с фильтрацией по вершинам на пути. \label{query5}
\end{enumerate}

Графовые запросы выполняются над четырьмя разнородными наборами данных в формате CSV. Далее излагаются описание наборов,
их структура, графовая интерпретация, а также конкретные схема и запросы DQL.

\subsection{Набор данных Elliptic++ Transactions}

\subsubsection{Описание и структура}

Набор Elliptic++ Transactions~\cite{elliptic} содержит данные о 203\,769 транзакциях Bitcoin, их легальности и потоке.
Набор состоит из следующих файлов:
\begin{itemize}
  \item файл \texttt{txs\_features.csv} (663 МБ), в котором определяются основные атрибуты транзакций: временн\'{a}я
    метка в целочисленном диапазоне [0, 49], 93 локальных атрибута, описывающих собственную информацию транзакции, 72
    агрегированных атрибута, составленных на основе входящих и исходящих транзакций, и 17 дополнительных атрибутов,
    отражающих статистическую информацию;
  \item файл \texttt{txs\_classes.csv} (2.3 МБ), в котором определяется класс транзакции. Множество всех транзакций
    разбивается на три класса: легальные (42\,019), нелегальные (4\,545) и неопределённые (157\,205);
  \item файл \texttt{txs\_edgelist.csv} (4.3 МБ), в котором хранятся 234\,355 упорядоченных пар идентификаторов
    транзакций, отражающих их поток.
\end{itemize}

\subsubsection{Интерпретация и схема}

Исходные данные естественным образом представляются ориентированным графом, вершины которого --- транзакции, а
направленные рёбра соответствуют потоку транзакций. В листинге \ref{lst:elliptic:schema} приводится фрагмент DQL-схемы
набора, где определены предикаты, использующиеся в дальнейшем при выполнении запросов. Исходные идентификаторы
транзакций представляются предикатом \texttt{id}. Все исходящие транзакции из данной транзакции представляются
предикатом \texttt{successors}.

\begin{listing}[!htb]
\caption{Фрагмент DQL-схемы набора данных Elliptic++ Transactions.}
\inputminted[frame=single,fontsize=\footnotesize,linenos,breaklines,xleftmargin=1.5em, breaksymbol=""]{text}
{lst/elliptic/schema}
\label{lst:elliptic:schema}
\end{listing}

\subsubsection{Запросы}

Выбор вершин графа и предикатов, участвующих в запросах, является важнейшей составляющей оценки. Во многих
ситуациях показательные результаты достигаются, когда в запросе участвуют вершины с наибольшей полустепенью захода или
исхода --- это нужным образом усложняет выполнение запросов. Для обнаружения таких вершин пишутся вспомогательные
запросы.

\paragraph{Запрос \ref{query1}}

В листинге \ref{lst:elliptic:most_successors_transaction} приводится вспомогательный запрос для определения транзакции
с наибольшим числом исходящих транзакций. Предикат \texttt{id} этой транзакции равен 2\,984\,918, а количество
исходящих транзакций --- 472.

\begin{listing}[!htb]
\caption{Запрос для определения транзакции с наибольшим числом исходящих транзакций.}
\inputminted[frame=single,fontsize=\footnotesize,linenos,breaklines,xleftmargin=1.5em, breaksymbol=""]{text}
{lst/elliptic/most_successors_transaction}
\label{lst:elliptic:most_successors_transaction}
\end{listing}

В листинге \ref{lst:elliptic:query1} приводится запрос \ref{query1}. Рекурсивный обход начинается в транзакции с
наибольшим числом исходящих транзакций и завершается на глубине 50. Вершины на пути фильтруются по медианной временной
метке.

\begin{listing}[!htb]
\caption{Запрос \ref{query1} для набора данных Elliptic++ Transactions.}
\inputminted[frame=single,fontsize=\footnotesize,linenos,breaklines,xleftmargin=1.5em, breaksymbol=""]{text}
{lst/elliptic/query1}
\label{lst:elliptic:query1}
\end{listing}

\paragraph{Запрос \ref{query2}}

В листинге \ref{lst:elliptic:query2} приводится запрос \ref{query2}. В запросе выбираются транзакции, легальность
которых не определена --- транзакции со значением 3 предиката \texttt{class}, составляющие 77\% от всех транзакций.

\begin{listing}[!htb]
\caption{Запрос \ref{query2} для набора данных Elliptic++ Transactions.}
\inputminted[frame=single,fontsize=\footnotesize,linenos,breaklines,xleftmargin=1.5em, breaksymbol=""]{text}
{lst/elliptic/query2}
\label{lst:elliptic:query2}
\end{listing}

\paragraph{Запрос \ref{query3}}

В листинге \ref{lst:elliptic:most_predecessors_transaction} приводится вспомогательный запрос для определения
транзакции с наибольшим числом входящих транзакций. Предикат \texttt{id} этой транзакции равен 43\,388\,675, а
количество входящих транзакций --- 284.

\begin{listing}[!htb]
\caption{Запрос для определения транзакции с наибольшим числом входящих транзакций.}
\inputminted[frame=single,fontsize=\footnotesize,linenos,breaklines,xleftmargin=1.5em, breaksymbol=""]{text}
{lst/elliptic/most_predecessors_transaction}
\label{lst:elliptic:most_predecessors_transaction}
\end{listing}

В листинге \ref{lst:elliptic:query3} приводится запрос \ref{query3}. Из транзакций запроса \ref{query2} отбираются те,
что входят в транзакцию, обладающую наибольшим числом входящих транзакций. Полученные транзакции фильтруются по степени
10.

\begin{listing}[!htb]
\caption{Запрос \ref{query3} для набора данных Elliptic++ Transactions.}
\inputminted[frame=single,fontsize=\footnotesize,linenos,breaklines,xleftmargin=1.5em, breaksymbol=""]{text}
{lst/elliptic/query3}
\label{lst:elliptic:query3}
\end{listing}

\paragraph{Запрос \ref{query4}}

В листинге \ref{lst:elliptic:query4} приводится запрос \ref{query4}. Для каждой транзакции подсчитывается среднее
значение поля \texttt{Local\_feature\_1} связанных транзакций, которые удовлетворяют фильтру на
\texttt{Local\_feature\_1}.

\begin{listing}[!htb]
\caption{Запрос \ref{query4} для набора данных Elliptic++ Transactions.}
\inputminted[frame=single,fontsize=\footnotesize,linenos,breaklines,xleftmargin=1.5em, breaksymbol=""]{text}
{lst/elliptic/query4}
\label{lst:elliptic:query4}
\end{listing}

\paragraph{Запрос \ref{query5}}

В листинге \ref{lst:elliptic:query5} приводится запрос \ref{query5}. Выполняется поик кратчайшего пути между
транзакциями с наибольшим числом исходящих и входящих транзакций. Транзакции на пути фильтруются по временной метке.

\begin{listing}[!htb]
\caption{Запрос \ref{query5} для набора данных Elliptic++ Transactions.}
\inputminted[frame=single,fontsize=\footnotesize,linenos,breaklines,xleftmargin=1.5em, breaksymbol=""]{text}
{lst/elliptic/query5}
\label{lst:elliptic:query5}
\end{listing}

\subsection{Набор данных MOOC User Actions}

\subsubsection{Описание и структура}

Набор MOOC User Actions~\cite{mooc} содержит данные о 411\,749 действиях 7\,047 пользователей на платформе онлайн-курсов
MOOC. Каждое действие ассоциировано с одним из 97 онлайн-курсов.

Набор состоит из следующих файлов:
\begin{itemize}
  \item файл \texttt{mooc\_actions.tsv} (11 МБ), в котором пользователи и курсы связываются отношением многие ко многим
    посредством действий, и действиям назначаются временные метки;
  \item файл \texttt{mooc\_action\_features.tsv} (35 МБ), в котором определяются 4 вещественнозначных атрибута действия;
  \item файл \texttt{mooc\_action\_labels.tsv} (3.5 МБ), в котором действиям назначаются бинарные метки соответственно
    тому, является ли действие последним действием пользователя на курсе.
\end{itemize}

\subsubsection{Интерпретация и схема}

Исходные данные представляются ориентированным графом, вершины которого --- пользователи, действия и курсы, а рёбра
связывают пользователей с действиями и действия с курсами. В листинге \ref{lst:mooc:schema} приводится DQL-схема набора.
Исходные идентификаторы пользователей, действий и курсов представляются предикатами \texttt{userId}, \texttt{actionId} и
\texttt{targetId} соответственно. Все действия одного пользователя выражаются предикатом \texttt{performs}, а связь
действия и курса описывается предикатом \texttt{on}.

\begin{listing}[!htb]
\caption{DQL-схема набора данных MOOC User Actions.}
\inputminted[frame=single,fontsize=\footnotesize,linenos,breaklines,xleftmargin=1.5em, breaksymbol=""]{text}
{lst/mooc/schema}
\label{lst:mooc:schema}
\end{listing}

\subsubsection{Запросы}

\paragraph{Запрос \ref{query1}}

В листинге \ref{lst:mooc:query1_aux} приводится вспомогательный запрос для определения двух пользователей, совершивших
наибольшее число действий. Ими являются пользователи с \texttt{userId} 1181 (505 действий) и \texttt{userId} 1686 (470
действий).

\begin{listing}[!htb]
\caption{Запрос для определения двух пользователей, совершивших наибольшее число действий.}
\inputminted[frame=single,fontsize=\footnotesize,linenos,breaklines,xleftmargin=1.5em, breaksymbol=""]{text}
{lst/mooc/query1_aux}
\label{lst:mooc:query1_aux}
\end{listing}

В листинге \ref{lst:mooc:query1} приводится запрос \ref{query1}. Рекурсивный обход начинается с пользователя,
совершившего наибольшее число действий, и завершается на глубине 5. В обходе участвуют предикаты \texttt{performs},
\texttt{on}, а также их обратные версии --- предикаты \texttt{\textasciitilde performs} и \texttt{\textasciitilde on}.
Это позволяет проводить обход графа без учёта направления рёбер. Фильтрация действий на пути проводится по временной
метке.

\begin{listing}[!htb]
\caption{Запрос \ref{query1} для набора данных MOOC User Actions.}
\inputminted[frame=single,fontsize=\footnotesize,linenos,breaklines,xleftmargin=1.5em, breaksymbol=""]{text}
{lst/mooc/query1}
\label{lst:mooc:query1}
\end{listing}

\paragraph{Запрос \ref{query2}}

В листинге \ref{lst:mooc:query2} приводится запрос \ref{query2}. В выборке участвуют действия, не являющиеся последними
действиями пользователя на курсе. Значение предиката \texttt{class} этих действий равно 0, а их количество составляет
95\% от количества всех действий.

\begin{listing}[!htb]
\caption{Запрос \ref{query2} для набора данных MOOC User Actions.}
\inputminted[frame=single,fontsize=\footnotesize,linenos,breaklines,xleftmargin=1.5em, breaksymbol=""]{text}
{lst/mooc/query2}
\label{lst:mooc:query2}
\end{listing}

\paragraph{Запрос \ref{query3}}

В листинге \ref{lst:mooc:query3_aux} приводится вспомогательный запрос для определения курса, с которым ассоциировано
наибольшее число действий --- курса с \texttt{targetId} 8 и 19\,474 действиями.

\begin{listing}[!htb]
\caption{Запрос для определения курса, с которым ассоциировано наибольшее число действий.}
\inputminted[frame=single,fontsize=\footnotesize,linenos,breaklines,xleftmargin=1.5em, breaksymbol=""]{text}
{lst/mooc/query3_aux}
\label{lst:mooc:query3_aux}
\end{listing}

В листинге \ref{lst:mooc:query3} приводится запрос \ref{query3}. Из множества действий запроса \ref{query2} выбираются
действия, относящиеся к курсу с \texttt{targetId} 8. Все вершины полученного множества имеют степень 2 (тривиально, по
своей структуре); соответствующая проверка добавлена для поддержания единообразия запросов.

\begin{listing}[!htb]
\caption{Запрос \ref{query3} для набора данных MOOC User Actions.}
\inputminted[frame=single,fontsize=\footnotesize,linenos,breaklines,xleftmargin=1.5em, breaksymbol=""]{text}
{lst/mooc/query3}
\label{lst:mooc:query3}
\end{listing}

\paragraph{Запрос \ref{query4}}

В листинге \ref{lst:mooc:query4} приводится запрос \ref{query4}. Для каждого пользователя и курса подсчитывается среднее
значение поля \texttt{feature1} действий, с которыми этот пользователь или курс связан. В подсчёте учитываются только
неотрицательные значения поля \texttt{feature1}.

\begin{listing}[!htb]
\caption{Запрос \ref{query4} для набора данных MOOC User Actions.}
\inputminted[frame=single,fontsize=\footnotesize,linenos,breaklines,xleftmargin=1.5em, breaksymbol=""]{text}
{lst/mooc/query4}
\label{lst:mooc:query4}
\end{listing}

\paragraph{Запрос \ref{query5}}

В листинге \ref{lst:mooc:query5} приводится запрос \ref{query5}. Кратчайший путь между пользователями с \texttt{userId}
1181 и \texttt{userId} 1686 вычисляется без учёта направления рёбер и с фильтрацией вершин на пути по временной метке.

\begin{listing}[!htb]
\caption{Запрос \ref{query5} для набора данных MOOC User Actions.}
\inputminted[frame=single,fontsize=\footnotesize,linenos,breaklines,xleftmargin=1.5em, breaksymbol=""]{text}
{lst/mooc/query5}
\label{lst:mooc:query5}
\end{listing}

\subsection{Набор данных California Road Network}

\subsubsection{Описание и структура}

Набор California Road Network~\cite{roadnet} содержит данные о дорожной сети Калифорнии. Набор содержит единственный
исходный файл \texttt{roadNet-CA.txt} (84 МБ), в котором определяются узлы дорожной сети --- пункты назначения или
пересечения дорог, и двусторонние связи между ними. Количество узлов: 1\,965\,206, и они не имеют атрибутов; количество
связей: 2\,766\,607.

\subsubsection{Интерпретация и схема}

Исходные данные представляются обыкновенным неориентированным графом. DQL-схема набор приводится в листинге
\ref{lst:roadnet:schema}. Предикат \texttt{id} представляет исходный идентификатор узла, а предикат \texttt{successors}
--- связанные узлы.

\begin{listing}[!htb]
\caption{DQL-схема набора данных California Road Network.}
\inputminted[frame=single,fontsize=\footnotesize,linenos,breaklines,xleftmargin=1.5em, breaksymbol=""]{text}
{lst/roadnet/schema}
\label{lst:roadnet:schema}
\end{listing}

\subsubsection{Запросы}

\paragraph{Запрос \ref{query1}}

В листинге \ref{lst:roadnet:query1_aux} приводится вспомогательный запрос для определения двух узлов с наибольшей
степенью. Такими являются узлы с \texttt{id} 562\,818 (степень 12) и \texttt{id} 521\,168 (степень 10).

\begin{listing}[!htb]
\caption{Запрос для определения двух узлов наибольшей степени.}
\inputminted[frame=single,fontsize=\footnotesize,linenos,breaklines,xleftmargin=1.5em, breaksymbol=""]{text}
{lst/roadnet/query1_aux}
\label{lst:roadnet:query1_aux}
\end{listing}

В листинге \ref{lst:roadnet:query1} приводится запрос \ref{query1}. Поскольку узлы не обладают нетривиальными
предикатами, для имитации этих предикатов используется \texttt{id}. Так, рекурсивный запрос начинается в узле с
\texttt{id} 562\,818, завершается на глубине 50, и вершины на пути фильтруются по значению предиката \texttt{id}.

\begin{listing}[!htb]
\caption{Запрос \ref{query1} для набора данных California Road Network.}
\inputminted[frame=single,fontsize=\footnotesize,linenos,breaklines,xleftmargin=1.5em, breaksymbol=""]{text}
{lst/roadnet/query1}
\label{lst:roadnet:query1}
\end{listing}

\paragraph{Запрос \ref{query2}}

В листинге \ref{lst:roadnet:query2} приводится запрос \ref{query2}. Выборка запроса тривиальная и совершается по
значению 562\,818 предиката \texttt{id}.

\begin{listing}[!htb]
\caption{Запрос \ref{query2} для набора данных California Road Network.}
\inputminted[frame=single,fontsize=\footnotesize,linenos,breaklines,xleftmargin=1.5em, breaksymbol=""]{text}
{lst/roadnet/query2}
\label{lst:roadnet:query2}
\end{listing}

\paragraph{Запрос \ref{query3}}

В листинге \ref{lst:roadnet:query3} приводится запрос \ref{query3}. Для узла с \texttt{id} 562\,818 проверяется наличие
связи с узлом, \texttt{id} которого равен 562\,826 (в действительности это непосредственный сосед исходного узла). Далее
проверяется, что степень узла совпадает с ожидаемой --- с 12.

\begin{listing}[!htb]
\caption{Запрос \ref{query3} для набора данных California Road Network.}
\inputminted[frame=single,fontsize=\footnotesize,linenos,breaklines,xleftmargin=1.5em, breaksymbol=""]{text}
{lst/roadnet/query3}
\label{lst:roadnet:query3}
\end{listing}

\paragraph{Запрос \ref{query4}}

В листинге \ref{lst:roadnet:query4} приводится запрос \ref{query4}. Для каждого узла вычисляется среднее арифметическое
значений \texttt{id} соседних узлов, удовлетворяющих фильтру.

\begin{listing}[!htb]
\caption{Запрос \ref{query4} для набора данных California Road Network.}
\inputminted[frame=single,fontsize=\footnotesize,linenos,breaklines,xleftmargin=1.5em, breaksymbol=""]{text}
{lst/roadnet/query4}
\label{lst:roadnet:query4}
\end{listing}

\paragraph{Запрос \ref{query5}}

В листинге \ref{lst:roadnet:query5} приводится запрос \ref{query5}. Поиск кратчайшего пути выполняется между узлами с
наибольшей степенью. Фильтрация производится по предикату \texttt{id}.

\begin{listing}[!htb]
\caption{Запрос \ref{query5} для набора данных California Road Network.}
\inputminted[frame=single,fontsize=\footnotesize,linenos,breaklines,xleftmargin=1.5em, breaksymbol=""]{text}
{lst/roadnet/query5}
\label{lst:roadnet:query5}
\end{listing}

\subsection{Набор данных Stablecoin ERC20 Transactions} \label{datasetERC20}

\subsubsection{Описание и структура}

\textit{Стейблкоинами} называют специальные токены, предназначенные для поддержания фиксированной стоимости в течение
долгого времени. Набор Stablecoin ERC20 Transactions~\cite{erc20} содержит данные о более чем 70 миллионах транзакций
ERC20 пяти популярных стейблкоинов. Описания транзакций распределены по трём файлам набора, имеющим одинаковую
структуру: \texttt{token\_transfers.csv} (823 МБ), \texttt{token\_transfers\_V2.0.0.csv} (4.4 ГБ) и
\texttt{token\_transfers\_V3.0.0.csv} (5.6 ГБ). В файлах определяются \textit{переводы} (transfers), совершаемые в
рамках транзакций. Описание каждого перевода включает номер блока транзакции, индекс транзакции, количество переданных
стейблкоинов, временную метку, а также адреса отправителя, получателя и контракта, определяющего стейблкоин. Набор
также содержит файл с описанием событий, повлиявших на работу сети, файлы с изменением стоимости стейблкоинов, однако
объём информации в них незначительный (суммарно, 60 КБ), и этими данными можно пренебречь. Далее работа ведётся только с
файлом \texttt{token\_transfers.csv}.

\subsubsection{Интерпретация и схема}

Исходные данные можно представить ориентированным графом. В отличие от набора данных Elliptic++ Transactions, где
транзакции являются единственными вершинами в графе и непосредственно связаны друг с другом, здесь вершины графа ---
переводы и адреса отправителей, получателей или контрактов. Таким образом, переводы являются посредниками при выражении
отношения многие ко многим между адресами отправителей и получателей. Направленные рёбра в графе связывают
\begin{itemize}
  \item адрес отправителя и перевод;
  \item перевод и адрес получателя;
  \item перевод и адрес контракта.
\end{itemize}
В листинге \ref{lst:erc20:schema} приводится DQL-схема набора. Исходные адреса отправителей, получателей и контрактов
представляются предикатом \texttt{address}. Все переводы некоторого отправителя выражаются предикатом \texttt{from}.
Связь перевода и получателя описывается предикатом \texttt{to}, и связь перевода и контракта --- предикатом
\texttt{contract}.

\begin{listing}[!htb]
\caption{DQL-схема набора данных Stablecoin ERC20 Transactions.}
\inputminted[frame=single,fontsize=\footnotesize,linenos,breaklines,xleftmargin=1.5em, breaksymbol=""]{text}
{lst/erc20/schema}
\label{lst:erc20:schema}
\end{listing}

\subsubsection{Запросы}

\paragraph{Запрос \ref{query1}}

В листинге \ref{lst:erc20:greatest_transfers} приводится вспомогательный запрос для определения адреса отправителя
наибольшего числа переводов. Соответствующий предикат \texttt{address} равен
\texttt{0x74de5d4fcbf63e00296fd95d33236b9794016631}, а количество переводов --- 147\,437.

\begin{listing}[!htb]
\caption{Запрос для определения адреса отправителя наибольшего числа переводов.}
\inputminted[frame=single,fontsize=\footnotesize,linenos,breaklines,xleftmargin=1.5em, breaksymbol=""]{text}
{lst/erc20/greatest_transfers}
\label{lst:erc20:greatest_transfers}
\end{listing}

В листинге \ref{lst:erc20:query1} приводится запрос \ref{query1}. Рекурсивный обход выполняется от адреса отправителя
наибольшего числа переводов и до глубины 3. На пути рассматриваются предикат \texttt{from} с фильтрацией по временной
метке и предикат \texttt{to}.

\begin{listing}[!htb]
\caption{Запрос \ref{query1} для набора данных Stablecoins ERC20 Transactions.}
\inputminted[frame=single,fontsize=\footnotesize,linenos,breaklines,xleftmargin=1.5em, breaksymbol=""]{text}
{lst/erc20/query1}
\label{lst:erc20:query1}
\end{listing}

\paragraph{Запрос \ref{query2}}

В листинге \ref{lst:erc20:query2} приводится запрос \ref{query2}. Переводы отбираются по числу стейблкоинов.

\begin{listing}[!htb]
\caption{Запрос \ref{query2} для набора данных Stablecoins ERC20 Transactions.}
\inputminted[frame=single,fontsize=\footnotesize,linenos,breaklines,xleftmargin=1.5em, breaksymbol=""]{text}
{lst/erc20/query2}
\label{lst:erc20:query2}
\end{listing}

\paragraph{Запрос \ref{query3}}

В листинге \ref{lst:erc20:most_transfers_contract} приводится вспомогательный запрос для определения адреса контракта с
наибольшим числом переводов.

\begin{listing}[!htb]
\caption{Запрос для определения адреса контракта с наибольшим числом переводов.}
\inputminted[frame=single,fontsize=\footnotesize,linenos,breaklines,xleftmargin=1.5em, breaksymbol=""]{text}
{lst/erc20/most_transfers_contract}
\label{lst:erc20:most_transfers_contract}
\end{listing}

В листинге \ref{lst:erc20:query3} приводится запрос \ref{query3}. Переводы запроса \ref{query2} фильтруются по
адресу контракта с наибольшим числом переводов. Полученные вершины фильтруются по степени 2.

\begin{listing}[!htb]
\caption{Запрос \ref{query3} для набора данных Stablecoins ERC20 Transactions.}
\inputminted[frame=single,fontsize=\footnotesize,linenos,breaklines,xleftmargin=1.5em, breaksymbol=""]{text}
{lst/erc20/query3}
\label{lst:erc20:query3}
\end{listing}

\paragraph{Запрос \ref{query4}}

В листинге \ref{lst:erc20:query4} приводится запрос \ref{query4}. Для каждого адреса подсчитывается среднее значение
стейблкоинов связанных переводов, которые удовлетворяют фильтру.

\begin{listing}[!htb]
\caption{Запрос \ref{query4} для набора данных Stablecoins ERC20 Transactions.}
\inputminted[frame=single,fontsize=\footnotesize,linenos,breaklines,xleftmargin=1.5em, breaksymbol=""]{text}
{lst/erc20/query4}
\label{lst:erc20:query4}
\end{listing}

\paragraph{Запрос \ref{query5}}

В листинге \ref{lst:erc20:query5} приводится запрос \ref{query5}. Выполняется поиск кратчайшего пути между адресом
отправителя наибольшего числа контрактов и адресом 0, регулярно участвующим в переводах. Вершины на пути фильтруются
по временной метке.

\begin{listing}[!htb]
\caption{Запрос \ref{query5} для набора данных Stablecoins ERC20 Transactions.}
\inputminted[frame=single,fontsize=\footnotesize,linenos,breaklines,xleftmargin=1.5em, breaksymbol=""]{text}
{lst/erc20/query5}
\label{lst:erc20:query5}
\end{listing}
%%%%%%%%%%%%%%%%%%%%%%%%%%%%%%%%%%%%%%%%%%%%%%%%%%%%%% TEXT WIDTH %%%%%%%%%%%%%%%%%%%%%%%%%%%%%%%%%%%%%%%%%%%%%%%%%%%%%%

% В конструкторском разделе на основе сделанного во введении обзора проводится обоснованный выбор
% предлагаемого алгоритма, подробно описывается его использование применительно к решаемой задаче.
% Также производится проектирование архитектуры и модулей разрабатываемого приложения или пакета
% программ, интерфейсов их взаимодействия, определяются форматы представления входных и выходных
% данных, разрабатываются структуры данных для их внутреннего представления. В данном разделе
% расчётно-пояснительной записки могут выполняться расчеты для определения объемов памяти,
% необходимой для хранения исходных данных, промежуточных и окончательных результатов, а также
% расчеты, позволяющие оценить теоретическую сложность реализуемых алгоритмов. Объем конструкторской
% части должен составлять 35-55% всего объема расчётно-пояснительной записки.

% Технологический раздел должен содержать обоснование выбранных для реализации комплекса программ
% языков программирования, технологий и сторонних библиотек. Для полноты изложения приводятся
% ключевые моменты программной реализации разрабатываемых программ. В технологический раздел также
% включаются руководство администратора и руководство пользователя. Руководство администратора
% должно включать описание процедуры инсталляции и деинсталляции приложения, параметры запуска из
% командной строки, перечень требований к аппаратному и системному программному обеспечению, наличию
% каких-либо сторонних приложений или библиотек. Руководство пользователя должно содержать подробное
% описание графического и/или командного интерфейса разработанного приложения, перечень сообщений об
% ошибках с их описанием. В случае разработки библиотеки функций, а не пользовательского или
% системного приложения, руководство пользователя заменяется на описание программного интерфейса
% (API) библиотеки. Если руководство администратора и/или руководство пользователя имеют
% значительный объем, они переносятся из технологического раздела расчётно-пояснительной записки в
% приложения. Объем технологического раздела расчётно-пояснительной записки составляет 35-40%.

%%%%%%%%%%%%%%%%%%%%%%%%%%%%%%%%%%%%%%%%%%%%%%%%%%%%%% TEXT WIDTH %%%%%%%%%%%%%%%%%%%%%%%%%%%%%%%%%%%%%%%%%%%%%%%%%%%%%%
\section{Загрузка данных и выполнение запросов}

\subsection{Развёртывание кластера Dgraph}

Для развёртывания кластера Dgraph используется инструмент Docker Compose~\cite{dockerCompose},
предназначенный для работы с мультиконтейнерными приложениями. Это позволяет развёртывать узлы
кластера Dgraph в изолированных процессах --- Docker-контейнерах, что не требует выполнения
предварительных работ на основном устройстве, а также позволяет единственный раз настроить работу контейнеров
и их взаимодействие между в конфигурационном файле Docker Compose. В листинге \ref{lst:compose} приложения
приводится конфигурационный файл \texttt{compose.yml} для кластера Dgraph.

Кластер Dgraph составляют один Zero-узел и один Alpha-узел, разделяющие единое хранилище данных
(Docker volume). Zero-узел использует следующие порты:
\begin{itemize}
    \item 5080 для внутрикластерного взаимодействия и работы с инструментами импорта данных Dgraph
(Live Loader, Bulk Loader) по протоколу gRPC;
    \item 6080 для администрирования кластером по протоколу HTTP.
\end{itemize}
Alpha-узел использует следующие порты:
\begin{itemize}
    \item 7080 для взаимодействия внутри кластера по протоколу gRPC;
    \item 8080 для обработки клиентских запросов по протоколу HTTP;
    \item 9080 для обработки клиентских запросов по протоколу gRPC.
\end{itemize}



\subsection{Преобразование исходных данных}

Импорт данных в новый кластер будет осуществляться с использование инструмента Dgraph Bulk Loader.
Bulk loader принимает данные в формате JSON или Dgraph-расширении формата RDF, однако исходные
датасеты представлены в формате CSV. Таким образом, возникают две задачи:
\begin{enumerate}
    \item непосредственная конвертация форматов данных;
    \item логическое преобразование данных из реляционной модели в графовую модель.
\end{enumerate}
В документации Dgraph для конвертации форматов CSV в JSON рекомендуется использовать инструмент
csv2json~\cite{csv2json}, что решает первую задачу. Для решения второй задачи предлагается передавать на вход
утилите jq~\cite{jq} результат конвертации, а также текст программы на высокоуровневом функциональном языке
программирования JQ, предназначенного для преобразования литералов JSON. Это решает задачу подготовки данных, но
требует написания индивидуальных, во многом дублирующих друг друга, программ на языке JQ для каждого датасета.
Более универсальным решением будет реализация собственного инструмента, способного по некоторому описанию исходных
файлов датасета и правил их преобразования генерировать данные в нужном виде. Поставим задачу разработки такого
инструмента.



\subsection{Проектирование преобразователя данных}


\subsubsection{Формат входных и выходных данных}

Для описания правил преобразования исходных файлов достаточно возможностей формата YAML ---
требуемое представляется на нём просто и естественно, синтаксис описания нетрудно корректируется, а
средства анализа YAML распространены. Обрабатываемые файлы датасета описываются в одном YAML-файле,
который далее мы будем называть \textit{конфигурационным}. Полная спецификация конфигурационного
файла приводится в разделе \ref{configSpecs}.

Чтобы упростить работу с определением расположения файлов, целесообразно хранить их в
структурированном виде. Определим, что исходные файлы датасета будут располагаться в одной
директории с именем \texttt{source}, а соответствующий конфигурационный файл \texttt{convert.yml}
--- находиться непосредственно рядом с \texttt{source}. Директорию, содержащую \texttt{source} и
\texttt{convert.yml}, будем называть \textit{корневой директорией датасета}.

Целевым форматом преобразования будет Dgraph-расширение формата RDF: оно выразительнее JSON,
поскольку позволяет при необходимости явно назначать объектам типы данных.
Результат преобразования будет сохраняться в файле \texttt{output.rdf} в корневой
директории датасета.

Таким образом, преобразователю достаточно будет передать путь к корневой директории датасета, где
впоследствии будет сгенерирован пригодный для импорта файл \texttt{output.rdf}. Удобной будет опция
передачи преобразователю пути к директории с несколькими корневым директориям сразу для параллельной
(при возможности) обработки соответствующих датасетов.


\subsubsection{Спецификация файла конфигурации} \label{configSpecs}

На верхнем уровне конфигурационного файла \texttt{convert.yml} должен быть определён массив
\texttt{files}, содержащий описания обрабатываемых файлов датасета. Незадействованные файлы
описывать в \texttt{files} не нужно. Файлы обрабатываются в порядке их следования в массиве
\texttt{files}. Ниже определяются допустимые поля в описании файлов. Все поля по умолчанию
считаются обязательными, если явно не оговорено обратное.

\paragraph{Поле \texttt{name}}

Определяет имя обрабатываемого файла.

\paragraph{Поле \texttt{delimiter}}

Определяет символ (единственный), разделяющий столбцы в файле. Поле опциональное, и по умолчанию
разделителем является запятая, что соответствует формату CSV. При работе с форматом TSV разделителем
является символ табуляции \texttt{\textbackslash t}. Символы возврата каретки \texttt{\textbackslash r}
или перевода строки \texttt{\textbackslash n} не могут быть разделителями.

\paragraph{Поле \texttt{comment}}

Определяет символ (единственный), являющийся началом однострочного комментария. Поле опциональное.
Все строки файла, начинающиеся непосредственно с \texttt{comment}, игнорируются. Если \texttt{comment}
стоит не в начале строки, он считается обычным символом. Символ начала комментария не может совпадать с
\texttt{delimiter} или быть равным \texttt{\textbackslash r}, \texttt{\textbackslash n}.

\paragraph{Поле \texttt{declarations}}

Массив с объявлениями имён и типов столбцов файла, используемых в правилах преобразования.
Каждое объявление содержит
\begin{itemize}
    \item поле \texttt{name} --- имя столбца;
    \item поле \texttt{type} --- тип (данных) стобца. Поддерживаемые типы: строка \texttt{string},
целое число \texttt{int} (32 бита), число с плавающей запятой \texttt{float} (64 бита) и
идентификатор \texttt{id}. Типы данных \texttt{string}, \texttt{int} и \texttt{float} также будут
называться \textit{тривиальными};
    \item поле \texttt{prefix} (устанавливается, только если \texttt{type} --- \texttt{id}). Если
атрибут определён, все значения столбца будут предварены указанным префиксом --- это необходимо для
обеспечения уникальности blank-идентификаторов. Действительно, поскольку идентификаторы в реляционных
таблицах обычно целочисленные, blank-идентификаторы c одинаковым номером, но соответствующие разным
сущностям, совпадут. Присоединение различных префиксов обеспечит blank-идентификаторам уникальность.
Как следствие, если одна и та же сущность используется в различных файлах датасета, префикс её
идентификаторов должен совпадать везде (или везде отсутствовать).
\end{itemize}
Подчеркнём, что все столбцы, используемые в правилах преобразования, должны быть определены в
массиве \texttt{declarations}. Незадействованные столбцы файла определять в \texttt{declarations}
не нужно.

\paragraph{Поле \texttt{artificial\_declaration}}

Объявление искусственного столбца, данные которого генерируются во время работы преобразователя. В данный
момент используется только для генерации искусственных идентификаторов, когда они явно не указываются в
исходных файлах (например в датасете \ref{datasetERC20} не предусмотрены идентификаторы передач в
\texttt{token\_transfers.csv}). Искусственное объявление содержит те же поля, что и обычное объявление:
\texttt{name}, \texttt{type} (только \texttt{id}) и \texttt{prefix}.

\paragraph{Поле \texttt{entity\_facets}}

Введём необходимый контекст. Сущность реляционной модели, хранящая информацию о сопоставлении других сущностей,
в графовой модели переходит в отношение между этими сущностями. Исходную сущность назовём \textit{соединительной}.
Атрибуты соединительной сущности переходят в атрибуты соответствующего отношения, или фасеты,
которые в Dgraph назначаются ребру все и в рамках одного объявления. Поскольку атрибуты соединительной сущности
могут располагаться в разных реляционных таблицах, обрабатывая файлы приходится <<накапливать>>
фасеты, ассоциированные с этой сущностью, до тех пор, пока они не будут собраны все. Только после
этого полученные фасеты можны добавить к фасетам выписываемого литерала RDF.

Итак, поле \texttt{entity\_facets} определяет массив правил, по которым в памяти преобразователя
сохраняются фасеты, ассоциированные с соединительной сущностью. Каждое такое правило содержит
\begin{itemize}
    \item поле \texttt{id} --- имя столбца идентификаторов соединительной сущности;
    \item поле \texttt{key} --- имя ключа фасета, которое будет использоваться в Dgraph;
    \item поле \texttt{value} --- имя столбца значений фасета. Столбец должен быть тривиального типа.
\end{itemize}

Таким образом, для каждого правила массива \texttt{enitity\_facets} со всяким идентификатором из
столбца \texttt{id} будет ассоциирован фасет с ключом \texttt{key} и соответствующим значением
столбца \texttt{value}. Сохранение фасетов происходит перед выписыванием литералов RDF.

\paragraph{Поле \texttt{rdfs}}

Массив правил, по которым в \texttt{output.rdf} выписываются литералы RDF. Каждое такое правило содержит
\begin{itemize}
    \item поле \texttt{subject} --- имя столбца субъекта RDF. Столбец должен быть типа \texttt{id};
    \item поле \texttt{predicat} --- имя соответствующего предиката схемы Dgraph;
    \item поле \texttt{object} --- имя столбца объекта RDF;
    \item опциональное поле \texttt{cast\_object\_to} --- изменённый тип объекта. Используется, если
необходимо выполнить преобразование типа объекта RDF (например, чтобы интерпретировать столбец типа \texttt{id}
как \texttt{int} для получения буквального значения идентификатора);
    \item опциональное поле \texttt{entity\_facets\_id} --- имя столбца идентификаторов
соединительной сущности. Применяется, если к фасетам данного литерала RDF необходимо добавить
фасеты, ассоциированные с соединительной сущностью.
    \item опциональное поле \texttt{facets} --- массив описаний фасетов, добавляемых к фасетам
данного литерала RDF. Отличается от \texttt{entity\_facets\_id} тем, что фасеты \texttt{facets}
не ассоциированы с какой-либо сущностью, а их значения берутся непосредственно из обрабатываемого
файла. Каждое описание фасета содержит
\begin{itemize}
    \item поле \texttt{key} --- имя ключа фасета, которое будет использоваться в Dgraph;
    \item поле \texttt{value} --- имя столбца значений фасета. Столбец должен быть тривиального
типа.
\end{itemize}
\end{itemize}


\subsubsection{Модули преобразователя}

При проектировании преобразователя естественным образом выделяются два модуля:
\begin{enumerate}
    \item Модуль работы с RDF. Содержит структуры данных для представления в программе литералов
Dgraph-расширения формата RDF, а также необходимые методы работы с ними.
    \item Модуль преобразования. Содержит структуры данных для представления конфигурационного
файла, методы валидации структур и всю основную функциональность преобразования.
\end{enumerate}

Для описания функциональности модулей частично используется объектно-ориентированная терминология:
понятия объекта, метода, конструктора и др. Говоря, что модуль <<предоставляет>> некоторое
определение, подразумевается, что оно публичное, т.е. может использоваться другими модулями.


\subsubsection{Модуль работы с RDF}

Определим представление субъекта, предиката и объекта RDF --- термов RDF --- в программе. Заметим,
что в терме можно выделить его значение, а также <<оформление>> этого значения, влияющее на
контекст использования. Например, blank-идентификатор указывается без оформления, UID обрамляется
треугольными скобками, а литеральное значение берётся в двойные кавычки.

Введём тип перечисления (enumeration) \texttt{Decoration} с допустимыми значениями \texttt{None},
\texttt{AngleBrackets} и \texttt{Quotes}. Тогда терм RDF будет представляться структурой
\texttt{Term} с полями
\begin{itemize}
    \item \texttt{value} --- значение терма;
    \item \texttt{decoration} --- оформление значения терма типа \texttt{Decoration}.
\end{itemize}

Значения фасетов RDF также могут иметь оформление. Например, целочисленный литерал ничем не
обрамляется, а строковый --- берётся в двойные кавычки. Соответственно, фасет RDF представляется
структурой \texttt{Facet} с полями
\begin{itemize}
    \item \texttt{key} --- имя ключа фасета;
    \item \texttt{value} --- значение фасета;
    \item \texttt{decoration} --- оформление значения фасета типа \texttt{Decoration}.
\end{itemize}

Структура \texttt{Rdf} представляет литерал RDF (с фасетами). \texttt{Rdf} содержит поля
\begin{itemize}
    \item \texttt{subject} --- субъект RDF типа \texttt{Term};
    \item \texttt{predicat} --- предикат RDF типа \texttt{Term};
    \item \texttt{object} --- объект RDF типа \texttt{Term};
    \item \texttt{facets} --- массив фасетов RDF типа \texttt{Facet}.
\end{itemize}

Модуль предоставляет тип \texttt{Decoration}, структуры \texttt{Term}, \texttt{Facet} и
\texttt{Rdf} и их поля, а также функции создания этих структур и методы их сериализации.


\subsubsection{Модуль преобразования}

Мы предполагаем, что для представления конфигурационного файла сущностями в программе используется
некоторый инструмент десериализации YAML. Поскольку исходный конфигурационный файл может содержать
ошибки, указанные сущности необходимо валидировать: проверять отсутствие необъявленных значений в
правилах, соответствие типов используемых столбцов контекстным ограничениям и т.п. Далее при
определении структур данных будем подразумевать, что для них определены методы валидации
в соответствии со спецификацией конфигурационного файла.

\paragraph{Представление правил}

Сначала определим структуры данных для представления правил сохранения фасетов и выписывания литералов
RDF при обработке некоторого исходного файла датасета.

Правило сохранения фасета, ассоциированного с соединительной сущностью, представляется структурой
\texttt{EntityFacetRule}. Структура содержит поля
\begin{itemize}
    \item \texttt{id} --- имя столбца идентификаторов соединительной сущности;
    \item \texttt{key} --- имя ключа фасета;
    \item \texttt{value} --- имя столбца значений фасета.
\end{itemize}

Правило выписывания литерала RDF может содержать описания фасетов, добавляемых к фасетам данного
литерала RDF, не ассоциированных с какой-либо соединительной сущностью, и значения которых
берутся непосредственно из текущего файла. Они представляются структурой \texttt{FacetRule} с полями
\begin{itemize}
    \item \texttt{key} --- имя ключа фасета;
    \item \texttt{value} --- имя столбца значений фасета.
\end{itemize}

Правило выписывания литерала RDF представляется структурой \texttt{RdfRule}. Стуктура содержит поля
\begin{itemize}
    \item \texttt{subject} --- имя столбца субъекта RDF;
    \item \texttt{predicat} --- имя предиката RDF;
    \item \texttt{object} --- имя столбца объекта RDF;
    \item \texttt{cast\_object\_to} --- изменённый тип объекта RDF;
    \item \texttt{facets} --- массив правил \texttt{FacetRule};
    \item \texttt{entityFacetsId} --- имя столбца идентификаторов соединительной сущности.
\end{itemize}

\paragraph{Обработка файлов}

Для представления допустимых типов столбцов файла вводится перечисление \texttt{DataType}, содержащее
элементы \texttt{String}, \texttt{Int}, \texttt{Float} и \texttt{Id}.

Объявление имени и типа столбца файла, используемое в правилах преобразования \texttt{FacetRule},
\texttt{EntityFacetRule} и \texttt{RdfRule}, представляется структурой \texttt{Declaration}.
Структура содержит поля
\begin{itemize}
    \item \texttt{name} --- имя столбца в файле;
    \item \texttt{type} --- тип столбца из \texttt{DataType});
    \item \texttt{prefix} --- префикс идентификатора.
\end{itemize}

Описание исходного файла датасета и правил его преобразования представляется структурой \texttt{File}.
Структура содержит поля
\begin{itemize}
    \item \texttt{name} --- имя файла;
    \item \texttt{delimiter} --- символ разделителя столбцов в файле;
    \item \texttt{comment} --- символ начала однострочного комментария;
    \item \texttt{declarations} --- массив объявлений типа \texttt{Declaration};
    \item \texttt{artificial\_declaraion} --- искусственное объявление типа \texttt{Declaration};
    \item \texttt{entityFacets} --- массив правил типа \texttt{EntityFacet};
    \item \texttt{rdfs} --- массив правил типа \texttt{RdfRule}.
\end{itemize}
Предполагается, что при обработке исходных файлов датасета используется некоторый инструмент
десериализации CSV.

Структура \texttt{Dataset} содержит данные верхнего уровня конфигурационного файла, т.е.
единственный массив \texttt{files} описаний исходных файлов \texttt{File} датасета. При обработке
\texttt{files} используется общая таблица фасетов \texttt{entitiesFacets}, ассоциированных с
соединительными сущностями. Результат преобразования всех файлов записывается в единый файл
\texttt{output.rdf}.

Модуль предоставляет функции \texttt{ProcessDataset} и \texttt{ProcessDatasets}, принимающие
пути к одной и к нескольким корневым директориям датасетов соответственно и выполняющие необходимые
преобразования.



\subsection{Реализация преобразователя данных}

Для реализации выбран язык программирования Go версии 1.22 по следующим причинам:
\begin{itemize}
    \item обладание всеми необходимыми механизмами абстракции и нативная поддержка многопоточности 
(последнее пригодится при обработке нескольких датасетов сразу);
    \item обильная стандартная библиотека, предоставляющая, в частности, удобные инструменты работы
с форматами YAML и CSV;
    \item компилируемость, что обеспечивает б\'{о}льшую скорость выполнения программ в сравнении с
интерпретируемыми языками. Быстродействие преобразователя существенно, поскольку он рассчитан
на работу с большими объёмами данных. 
\end{itemize}


\subsubsection{Модуль работы с RDF}

Модуль работы с RDF реализован в Go-пакете \texttt{rdf}. Пакет предоставляет тип \texttt{Decoration},
структуры \texttt{Term}, \texttt{Facet}, \texttt{Rdf}, функции \texttt{NewTerm}, \texttt{NewFacet},
\texttt{NewRdf} конструирования структур и методы \texttt{String} их сериализации
(листинги \ref{lst:rdf:term}, \ref{lst:rdf:facet}, \ref{lst:rdf:rdf} приложения соответственно).


\subsubsection{Модуль преобразования}

Модуль преобразования реализован в \texttt{Go-пакете} \texttt{converter}. Опишем основные
особенности реализации модуля.

\paragraph{Обработка ошибок}

В реализации большое внимание уделяется валидации структур данных и обработке ошибок. При возникновении
ошибки соответствующий объект \texttt{error} передаётся среди возвращаемых значений задействованных функций,
оборачивая информацию об ошибке необходимым контекстом вплоть до вызова функций \texttt{ProcessDataset} и
\texttt{ProcessDatasets}.

\paragraph{Публичные функции}

Реализация функции \texttt{ProcessDatasets} приводится в листинге \ref{lst:converter:processDatasets} приложения.
Функция принимает путь к директории с корневыми директориями датасетов, и для каждой директории вызывает метод
\texttt{ProcessDataset}. Вызовы происходят в отдельных горутинах с использованием механизма синхронизации
\texttt{errgroup} стандартного пакета \texttt{sync} языка Go. Если при обработке какого-либо датасета возникает
ошибка, соответствующая горутина завершает свою работу, не влияя на работу остальных горутин, и функция
\texttt{ProcessDatasets} возвращает объект ошибки. В противном случае функция возвращает \texttt{nil}.

Реализация функции \texttt{ProcessDataset} приводится в листинге \ref{lst:converter:processDataset}
приложения. Функция принимает путь к корневой директории датасета и средствами стандартного пакета
\texttt{yaml.v3} языка Go пробует десериализовать конфигурационный файл \texttt{convert.yml} во внутренние
структуры модуля: \texttt{Dataset}, \texttt{File}, \texttt{Declaration} и пр. В случае успеха функция
передаёт управление обработчику (методу обработки) полученной структуры \texttt{Dataset}. При возникновении
ошибки функция возвращает соответветствующий объект, а иначе --- \texttt{nil}.

\paragraph{Внутренняя логика}

Обработчик структуры \texttt{Dataset} (листинг \ref{lst:converter:dataset.process} приложения) создаёт файл
\texttt{output.rdf}, а также таблицу фасетов \texttt{entitiesFacets} на основе стандартной хеш-таблицы
\texttt{map} языка Go; \texttt{output.rdf} и \texttt{entitiesFacets} разделяются обработчиками всех объектов
\texttt{files} датасета.

Обработчик структуры \texttt{File} (листинг \ref{lst:converter:file.process} приложения) вначале валидирует
\texttt{delimiter}, \texttt{comment} и преобразует \texttt{declarations}, \texttt{artificial\_declaration} к
удобному для работы виду. Для десериализации исходного файла используется стандартный пакет \texttt{encoding/csv}
языка Go. Для каждой полученной записи вызываются функции применения правил сохранения фасетов (листинг
\ref{lst:converter:saveFacets} приложения) и выписывания литералов RDF (листинг \ref{lst:converter:writeRdfs}
приложения). 


\subsubsection{Использование преобразователя}

Весь исходный код преобразователя находится в Git-репозитории~\cite{sources}. Для использования
преобразователя на Linux можно клонировать указанный репозиторий CLI-утилитой git и
применить CLI-утилиту go, предназначенную для работы с исходным кодом проектов на языке Go.
Программа преобразования \texttt{convert.go} располагается в директории \texttt{cmd/converter}.

Допустимые опции программы:
\begin{itemize}
    \item \texttt{dataset-path} --- путь к корневой директории датасета;
    \item \texttt{datasets-path} --- путь к директории с корневыми директориями датасетов;
    \item \texttt{help} --- вывод опций программы.
\end{itemize}
Одновременно может быть установлена только одна из опций \texttt{dataset-path} и \texttt{datasets-path}.



\subsection{Применение преобразователя}

Приведём конфигурационные файлы \texttt{convert.yml} преобразования датасетов. В листинге \ref{lst:elliptic:config}
приложения приводится фрагмент конфигурационного файла датасета Elliptic++ Transactions, в листинге
\ref{lst:mooc:config} приложения --- фрагмент конфигурационного файла датасета MOOC User Actions, в листинге
\ref{lst:roadnet:config} приложения --- конфигурационный файл датасета California Road Network и в листинге
\ref{lst:erc20:config} приложения --- фрагмент конфигурационного файла датасета Stablecoin ERC20 Transactions.



\subsection{Выполнение запросов}

Для выполнения запросов Dgraph и оценки ресурсов, затраченных на их выполнение, написана вспомогательная
программа. Она значительно проще, чем программа преобразования, поэтому можно сразу перейти к её реализации.


\subsubsection{Реализация}

Языком реализации также является Go, поскольку Dgraph предоставляет официальный Go-клиент
dgo~\cite{dgo}, который взаимодействует с сервером Dgraph по протоколу gRPC. Этот пакет и будет
использоваться при выполнении запросов, поскольку предоставляет всю необходимую функциональность.

\paragraph{Установка соединения}

Программа устанавливает соединение с сервером Dgraph и возвращает клиент Dgraph; соответствующая функция
приводится в листинге \ref{lst:benchmark:dial} приложения.

\paragraph{Выполнение транзакции}

В листинге \ref{lst:benchmark:perform} приложения приводится функция выполнения запроса и оценки времени,
оперативной памяти, затраченных на его обработку. Поскольку запросы в рамках поставленной задачи только
запрашивают данные и не
модифицируют их, для выполнения запросов клиентом Dgraph создаётся транзакция, предназначенная только для чтения
(read-only). Read-only транзакции могут обрабатываться в обход общего протокола выполнения запроса и повысить
скорость чтения.

\paragraph{Затраченное время}

Результат выполнения запроса содержит информацию о времени, затраченном на его обработку. А именно, это
время (в наносекундах):
\begin{itemize}
    \item синтаксического разбора текста запроса;
    \item непосредственно выполнения запроса;
    \item сериализации результата выполнения запроса;
\end{itemize}
Общее время выполнения запроса получается как сумма указанных значений.

\paragraph{Затраченная оперативная память}

Dgraph не предоставляет информации о потреблении оперативной памяти при выполнении запроса. Чтобы
получить эти данные, в отдельном потоке при выполнении запроса с постоянным интервалом во времени
отслеживается свободная оперативная память и, в частности, её минимальное значение. Разница между
свободной памятью до выполнения запроса и её минимальным значением во время выполнения интерпретируется
как потребление оперативной памяти. Разумеется, для получения корректных результатов на основном устройстве
не должны одновременно выполняться другие процессы, много и (или) непредсказуемо потребляющие
оперативную память. Для получения информации о свободной оперативной памяти в программе используется внешний
пакет \texttt{pbnjay/memory}~\cite{pbnjayMemory}.

\subsubsection{Использование}

Исходный код программы также находится в git-репозитории~\cite{sources}, в каталоге \texttt{cmd/benchmark}.
Доступные опции программы:
\begin{itemize}
    \item \texttt{query-path} --- путь к read-only запросу DQL;
    \item \texttt{host} --- хост сервера Dgraph (по умолчанию, \texttt{localhost});
    \item \texttt{port} --- gRPC-порт сервера Dgraph (по умолчнанию, \texttt{9080});
    \item \texttt{duration} --- временной интервал между замерами свободной оперативной памяти при
выполнении запроса в формате аргумента функции \texttt{time.ParseDuration} стандартной библиотеки
языка Go (по умолчанию, \texttt{100ms});
    \item \texttt{print-respond} --- печатать ли результат выполнения запроса в формате JSON;
    \item \texttt{help} --- вывод доступных опций программы.
\end{itemize}

%%%%%%%%%%%%%%%%%%%%%%%%%%%%%%%%%%%%%%%%%%%% TEXT WIDTH %%%%%%%%%%%%%%%%%%%%%%%%%%%%%%%%%%%%%%%%%%%%

% Раздел тестирования, во первых, должен содержать разработку и выполнение тестов, подтверждающих
% работоспособность созданного программного обеспечения. Во-вторых, в нем должны быть приведены
% результаты теоретического или экспериментального исследования, выполненного в ходе курсового
% проектирования. Например, это могут быть результаты, полученные при исследовании математического
% метода, положенного в основу разработанного алгоритма, или оценка временных и иных характеристик
% комплекса программ (алгоритма) в зависимости от особенностей входных данных. Объем раздела
% тестирования составляет 15-20%.

%%%%%%%%%%%%%%%%%%%%%%%%%%%%%%%%%%%%%%%%%%%% TEXT WIDTH %%%%%%%%%%%%%%%%%%%%%%%%%%%%%%%%%%%%%%%%%%%%
\section{Оценка производительности}

\subsection{Статистика датасетов}

Прежде чем представить результаты выполнения запросов, приведём сводную статистику по датасетам.

\subsubsection{Вершины и рёбра}

В таблице \ref{table:datasetsTopology} содержится итоговая информация о вершинах и рёберах датасетов.
Для датасета Elliptic++ Transactions эти данные остались прежними: число вершин есть число транзакций,
число рёбер --- число потоков транзакций. Для датасета MOOC User Actions число вершин складывается из
числа пользователей, действий и онлайн-курсов, а число рёбер равняется числу действий, домноженному на 2,
поскольку каждое действие соединяет одного пользователя и один курс. Для датасета California Road Network
данные также не изменились: число вершин --- число узлов, число рёбер --- число связей. Для датасета
Stablecoin ERC20 Transactions число вершин складывается из числа адресов, полученного вспомогательным
запросом, и числа передач; число рёбер есть трижды число передач, поскольку одна передача связывает
адрес отправителя, адрес получателя и адрес контракта.

\begin{table}[htb]
\caption{\centering Вершины и рёбра датасетов.}
\small
\centering\begin{tabular}{||c||c|c||}
\hline\hline
Датасет & Вершины & Рёбра \\
\hline\hline
Elliptic++ Transactions & 203\,769 & 234\,355 \\
\hline
MOOC User Actions & 418\,893 & 823\,498 \\
\hline
California Road Network & 1\,965\,206 & 2\,766\,607 \\
\hline
Stablecoin ERC20 Transactions & 6\,803\,466 & 15\,840\,393 \\ % 5 280 132 + 1 523 334
\hline\hline
\end{tabular}
\label{table:datasetsTopology}
\end{table}

\subsubsection{Дисковое пространство}

В таблице \ref{table:datasetsMemory} содержатся данные (в мегабайтах) о занимаемом дисковом
пространстве исходными файлами датасетов и данными кластера Dgraph после импорта датасетов.
Данные для исходных файлов тривиально получены суммированием размера файлов. Данные кластера
сняты с директории \texttt{dgraph/p} Docker-контейнера, который разделяют Alpha- и Zero-узлы
кластера.

\begin{table}[htb]
\caption{\centering Занимаемое датасетами дисковое пространство, МБ.}
\small
\centering\begin{tabular}{||c||c|c||}
\hline\hline
Датасет & Исходные файлы & Данные кластера \\
\hline\hline
Elliptic++ Transactions & 670 & 744 \\
\hline
MOOC User Actions & 48 & 74 \\
\hline
California Road Network & 84 & 122 \\
\hline
Stablecoin ERC20 Transactions & 823 & 901 \\
\hline\hline
\end{tabular}
\label{table:datasetsMemory}
\end{table}

\subsection{Выполнение запросов}

Для получения показателей следующих разделов все запросы выполнялись в одинаковых условиях и по 5 раз;
среднее значение результатов выполнения считается итоговым значением.

Характеристики устройства, на котором производилось оценивание:
\begin{itemize}
    \item объём оперативной памяти: 16 ГБ;
    \item процессор: Intel Core i5 10300H, 2500 МГц;
    \item видеокарта: Nvidia GeForce GTX 1650 Ti.
\end{itemize}

\subsubsection{Время выполнения}

В таблице \ref{table:queryTime} содержатся данные о времени выполнения запросов (в миллисекундах).
Под временем понимается общее время выполнения запроса, включающее синтаксический разбор текста
запроса, непосредственно выполнение и сериализацию результатов выполнения.

\begin{table}[htb]
\caption{\centering Время выполнения запросов, мс.}
\small
\centering\begin{tabular}{||c||c|c|c|c|c||}
\hline\hline
\backslashbox{Датасет}{Запрос} & \ref{query1} & \ref{query2} & \ref{query3} & \ref{query4} & \ref{query5} \\
\hline\hline
Elliptic++ Transactions & 27 & 85 & 448 & 3\,072 & 23 \\
\hline
MOOC User Actions & 8\,030 & 202 & 1\,235 & 2989 & 2\,667 \\
\hline
California Road Network & 8\,849 & 1 & 1 & 13\,661 & 30\,540 \\
\hline
Stablecoin ERC20 Transactions & 4\,199 & 1\,093 & 1\,099 & 56\,025 & 5\,546 \\
\hline\hline
\end{tabular}
\label{table:queryTime}
\end{table}

\subsubsection{Оперативная память}

В таблице \ref{table:queryMemory} содержатся данные о потреблении оперативной памяти
(в мегабайтах) во время выполнения запросов. Показатели получены, как разность свободной
оперативной памяти до выполнения запроса и её минимального значения во время выполнения.

\begin{table}[htb]
\caption{\centering Потребление ОЗУ при выполнении запросов, МБ.}
\small
\centering\begin{tabular}{||c||c|c|c|c|c||}
\hline\hline
\backslashbox{Датасет}{Запрос} & \ref{query1} & \ref{query2} & \ref{query3} & \ref{query4} & \ref{query5} \\
\hline\hline
Elliptic++ Transactions & 20 & 60 & 684 & 1\,941 & 19 \\
\hline
MOOC User Actions & 4\,340 & 112 & 1\,655 & 853 & 1\,839 \\
\hline
California Road Network & 419 & 4 & 1 & 9\,266 & 416 \\
\hline
Stablecoin ERC20 Transactions & 2\,447 & 351 & 954 & 14\,293 & 2\,170 \\
\hline\hline
\end{tabular}
\label{table:queryMemory}
\end{table}

%%%%%%%%%%%%%%%%%%%%%%%%%%%%%%%%%%%%%%%%%%%% TEXT WIDTH %%%%%%%%%%%%%%%%%%%%%%%%%%%%%%%%%%%%%%%%%%%%

%%%%%%%%%%%%%%%%%%%%%%%%%%%%%%%%%%%%%%%%%%%%%%%%%%%%%% TEXT WIDTH %%%%%%%%%%%%%%%%%%%%%%%%%%%%%%%%%%%%%%%%%%%%%%%%%%%%%%
\anonsection{ЗАКЛЮЧЕНИЕ}

В результате выполнения курсовой работы поставленные цели были достигнуты.

Важную и большую часть работы занимает программа преобразования, которая по специальному конфигурационному файлу
преобразует исходные файлы CSV-датасетов в расширение формата RDF, пригодное для импорта в Dgraph. Разработанный
инструмент позволяет унифицировать процесс преобразования форматов и не писать дублированный ad-hoc код. Выстроенная
архитектура преобразователя позволяет легко добавлять новые возможности, если это необходимо.

Выполнение запросов и их оценивание также удалось автоматизировать и унифицировать путём написания вспомогательной
программы. Полученные результаты могут быть воспроизведены и использованы для сравнения Dgraph с другими графовыми СУБД.
Ограниченная выразительность языка DQL, с одной стороны, позволяет эффективно обрабатывать определённый класс запросов
--- OLTP, а с другой --- делает СУБД неприменимой, если в работе применяются сложные алгоритмы на графах.
%%%%%%%%%%%%%%%%%%%%%%%%%%%%%%%%%%%%%%%%%%%%%%%%%%%%%% TEXT WIDTH %%%%%%%%%%%%%%%%%%%%%%%%%%%%%%%%%%%%%%%%%%%%%%%%%%%%%%


\renewcommand\refname{СПИСОК ИСПОЛЬЗОВАННЫХ ИСТОЧНИКОВ}
% Список литературы
\clearpage
%\bibliographystyle{ugost2008s}  %utf8gost71u.bst} %utf8gost705u} %gost2008s}
{\catcode`"\active\def"{\relax}
\addcontentsline{toc}{section}{\protect\numberline{}\refname}%
%\bibliography{biblio} %здесь ничего не меняем, кроме, возможно, имени bib-файла
\printbibliography
}
\newpage
\settocdepth{section}
\anonsection{ПРИЛОЖЕНИЕ}

\begin{listing}[!htb]
\caption{Конфигурационный файл Docker Compose кластера Dgraph}
\inputminted[frame=single,fontsize=\footnotesize,linenos,breaklines,xleftmargin=1.5em, breaksymbol=""]{yaml}
{lst/compose}
\label{lst:compose}
\end{listing}

%%% 

\begin{listing}[!htb]
\caption{Структура данных \texttt{Term} пакета \texttt{rdf} и её методы.}
\inputminted[frame=single,fontsize=\footnotesize,linenos,breaklines,xleftmargin=1.5em, breaksymbol=""]{go}
{lst/rdf/term}
\label{lst:rdf:term}
\end{listing}

\begin{listing}[!htb]
\caption{Структура данных \texttt{Facet} пакета \texttt{rdf} и её методы.}
\inputminted[frame=single,fontsize=\footnotesize,linenos,breaklines,xleftmargin=1.5em, breaksymbol=""]{go}
{lst/rdf/facet}
\label{lst:rdf:facet}
\end{listing}

\begin{listing}[!htb]
\caption{Структура данных \texttt{Rdf} пакета \texttt{rdf} и её методы.}
\inputminted[frame=single,fontsize=\footnotesize,linenos,breaklines,xleftmargin=1.5em, breaksymbol=""]{go}
{lst/rdf/rdf}
\label{lst:rdf:rdf}
\end{listing}

%%%

\begin{listing}[!htb]
\caption{Функция \texttt{ProcessDatasets} пакета \texttt{converter}.}
\inputminted[frame=single,fontsize=\footnotesize,linenos,breaklines,xleftmargin=1.5em, breaksymbol=""]{go}
{lst/converter/processDatasets}
\label{lst:converter:processDatasets}
\end{listing}

\begin{listing}[!htb]
\caption{Функция \texttt{ProcessDataset} пакета \texttt{converter}.}
\inputminted[frame=single,fontsize=\footnotesize,linenos,breaklines,xleftmargin=1.5em, breaksymbol=""]{go}
{lst/converter/processDataset}
\label{lst:converter:processDataset}
\end{listing}

\begin{listing}[!htb]
\caption{Метод \texttt{process} структуры \texttt{dataset} пакета \texttt{converter}.}
\inputminted[frame=single,fontsize=\footnotesize,linenos,breaklines,xleftmargin=1.5em, breaksymbol=""]{go}
{lst/converter/dataset.process}
\label{lst:converter:dataset.process}
\end{listing}

\begin{listing}[!htb]
\caption{Метод \texttt{process} структуры \texttt{file} пакета \texttt{converter}.}
\inputminted[frame=single,fontsize=\footnotesize,linenos,breaklines,xleftmargin=1.5em, breaksymbol=""]{go}
{lst/converter/file.process}
\label{lst:converter:file.process}
\end{listing}

\begin{listing}[!htb]
\caption{Функция \texttt{saveFacets} пакета \texttt{converter}.}
\inputminted[frame=single,fontsize=\footnotesize,linenos,breaklines,xleftmargin=1.5em, breaksymbol=""]{go}
{lst/converter/saveFacets}
\label{lst:converter:saveFacets}
\end{listing}

\begin{listing}[!htb]
\caption{Фрагмент функции \texttt{writeRdfs} пакета \texttt{converter}.}
\inputminted[frame=single,fontsize=\footnotesize,linenos,breaklines,xleftmargin=1.5em, breaksymbol=""]{go}
{lst/converter/writeRdfs}
\label{lst:converter:writeRdfs}
\end{listing}

%%%

\begin{listing}[!htb]
\caption{Фрагмент конфигурационного файла датасета Elliptic++ Transactions.}
\inputminted[frame=single,fontsize=\footnotesize,linenos,breaklines,xleftmargin=1.5em, breaksymbol=""]{text}
{lst/elliptic/config}
\label{lst:elliptic:config}
\end{listing}

\begin{listing}[!htb]
\caption{Фрагмент конфигурационного файла датасета MOOC User Actions.}
\inputminted[frame=single,fontsize=\footnotesize,linenos,breaklines,xleftmargin=1.5em, breaksymbol=""]{text}
{lst/mooc/config}
\label{lst:mooc:config}
\end{listing}

\begin{listing}[!htb]
\caption{Конфигурационный файл датасета California Road Network.}
\inputminted[frame=single,fontsize=\footnotesize,linenos,breaklines,xleftmargin=1.5em, breaksymbol=""]{text}
{lst/roadnet/config}
\label{lst:roadnet:config}
\end{listing}

\begin{listing}[!htb]
\caption{Фрагмент конфигурационного файла датасета Stablecoin ERC20 Transactions.}
\inputminted[frame=single,fontsize=\footnotesize,linenos,breaklines,xleftmargin=1.5em, breaksymbol=""]{text}
{lst/erc20/config}
\label{lst:erc20:config}
\end{listing}

%%%

\begin{listing}[!htb]
\caption{Функция формирования соединения с сервером Dgraph и получения клиента.}
\inputminted[frame=single,fontsize=\footnotesize,linenos,breaklines,xleftmargin=1.5em, breaksymbol=""]{go}
{lst/benchmark/dial}
\label{lst:benchmark:dial}
\end{listing}

\begin{listing}[!htb]
\caption{Функция выполнения запроса и оценки ресурсов, затраченных на его выполнение.}
\inputminted[frame=single,fontsize=\footnotesize,linenos,breaklines,xleftmargin=1.5em, breaksymbol=""]{go}
{lst/benchmark/perform}
\label{lst:benchmark:perform}
\end{listing}

\end{document}
