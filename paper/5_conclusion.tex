%%%%%%%%%%%%%%%%%%%%%%%%%%%%%%%%%%%%%%%%%%%%%%%%%%%%%% TEXT WIDTH %%%%%%%%%%%%%%%%%%%%%%%%%%%%%%%%%%%%%%%%%%%%%%%%%%%%%%
\anonsection{ЗАКЛЮЧЕНИЕ}

В результате выполнения курсовой работы поставленные цели были достигнуты.

Важную и большую часть работы занимает программа преобразования, которая по специальному конфигурационному файлу
преобразует исходные файлы CSV-датасетов в расширение формата RDF, пригодное для импорта в Dgraph. Разработанный
инструмент позволяет унифицировать процесс преобразования форматов и не писать дублированный ad-hoc код. Выстроенная
архитектура преобразователя позволяет легко добавлять новые возможности, если это необходимо.

Выполнение запросов и их оценивание также удалось автоматизировать и унифицировать путём написания вспомогательной
программы. Полученные результаты могут быть воспроизведены и использованы для сравнения Dgraph с другими графовыми СУБД.
Ограниченная выразительность языка DQL, с одной стороны, позволяет эффективно обрабатывать определённый класс запросов
--- OLTP, а с другой --- делает СУБД неприменимой, если в работе применяются сложные алгоритмы на графах.

%%%%%%%%%%%%%%%%%%%%%%%%%%%%%%%%%%%%%%%%%%%%%%%%%%%%%% TEXT WIDTH %%%%%%%%%%%%%%%%%%%%%%%%%%%%%%%%%%%%%%%%%%%%%%%%%%%%%%
