%%%%%%%%%%%%%%%%%%%%%%%%%%%%%%%%%%%%%%%%%%%%%%%%%%%%%%%%%%%%%%%%%%%%%%%%%%%%%%%%%%%%%%%%%%%%%%%%%%%%%%%%%%%%%%%%%%%%%%%%
\anonsection{ЗАКЛЮЧЕНИЕ}

В результате выполнения курсовой работы поставленные цели во многом были достигнуты.

Важную и большую часть работы занимает программа преобразования, которая по специальному конфигурационному файлу
преобразует исходные файлы наборов данных CSV в расширение формата RDF, пригодное для импорта в Dgraph. Разработанный
инструмент позволяет унифицировать процесс преобразования форматов и не писать дублированный ad hoc код. Выстроенная
архитектура преобразователя позволяет легко добавлять новые возможности, если это необходимо.

Выполнение запросов и их оценивание также удалось автоматизировать и унифицировать путём написания вспомогательной
программы. Полученные результаты могут быть воспроизведены и использованы для сравнения Dgraph с другими графовыми СУБД.
Ограниченная выразительность языка DQL, с одной стороны, позволяет эффективно обрабатывать определённый класс запросов
--- OLTP, а с другой --- делает СУБД неприменимой, если в работе используются сложные алгоритмы на графах.

Тем не менее, в оценке не учитываются некоторые возможности Dgraph, характеризующие его как распределённую графовую СУБД.

\begin{enumerate}
\item Тестовые данные интерпретируются без использования фасетов. Например, для наборов MOOC User Actions или Stablecoin
ERC20 Transactions возможно интерпретировать атрибуты пользовательских действий и переводов транзакций соответственно как
атрибуты отношений, а не вершин-посредников. Однако такая интерпретация хуже согласуется с определёнными графовыми
запросами, поскольку они в большинстве используют информацию о вершинах графа, а не отношениях.
\item Оценка произведена для кластера Dgraph, который составляют один Zero- и один Alpha-узел, что не отражает всех
возможностей масштабирования и распределённого использования СУБД. Это сделано также для возможности сравнения Dgraph
с другими СУБД в рамках курсового проекта.
\end{enumerate}
%%%%%%%%%%%%%%%%%%%%%%%%%%%%%%%%%%%%%%%%%%%%%%%%%%%%%%%%%%%%%%%%%%%%%%%%%%%%%%%%%%%%%%%%%%%%%%%%%%%%%%%%%%%%%%%%%%%%%%%%
