%%%%%%%%%%%%%%%%%%%%%%%%%%%%%%%%%%%%%%%%%%%%%%%%%%%%%%%%%%%%%%%%%%%%%%%%%%%%%%%%%%%%%%%%%%%%%%%%%%%%%%%%%%%%%%%%%%%%%%%%
\section{Оценка производительности}

\subsection{Статистика тестовых наборов данных}

\subsubsection{Вершины и рёбра}

В таблице \ref{table:datasetsTopology} содержится итоговая информация о числе вершин и рёбер тестовых наборов данных.
Для набора Elliptic++ Transactions эти данные остались прежними: число вершин есть число транзакций, число рёбер ---
число потоков транзакций. Для набора MOOC User Actions число вершин складывается из числа пользователей, действий и
онлайн-курсов, а число рёбер равняется удвоенному числу действий, поскольку одно действие соединяет пользователя и
онлайн-курс. Для набора California Road Network данные также не изменились: число вершин --- число узлов, число рёбер
--- число связей. Для набора Stablecoin ERC20 Transactions число вершин складывается из числа адресов, полученного
вспомогательным запросом, и числа переводов; число рёбер равняется утроенному числу переводов, поскольку один перевод
связывает адрес отправителя, адрес получателя и адрес контракта.

\begin{table}[!htb]
\caption{\centering Число вершин и рёбер тестовых наборов данных.}
\small
\centering\begin{tabular}{||c||c|c||}
\hline\hline
Набор данных & Число вершин & Число рёбер \\
\hline\hline
Elliptic++ Transactions & 203\,769 & 234\,355 \\
\hline
MOOC User Actions & 418\,893 & 823\,498 \\
\hline
California Road Network & 1\,965\,206 & 2\,766\,607 \\
\hline
Stablecoin ERC20 Transactions & 6\,803\,466 & 15\,840\,393 \\
\hline\hline
\end{tabular}
\label{table:datasetsTopology}
\end{table}

\subsubsection{Дисковое пространство}

В таблице \ref{table:datasetsMemory} содержится информация о дисковом пространстве, занимаемом исходными файлами
тестовых наборов и данными кластера Dgraph. Показатели для исходных файлов тривиально получены суммированием размера
файлов. Показатели для кластера Dgraph сняты с директории \texttt{/dgraph/p} хранилища Docker, который разделяют
Alpha- и Zero-узлы кластера. Директория \texttt{/dgraph/p} содержит загруженные данные, индексы на предикаты и обратные
рёбра во внутреннем представлении Dgraph.

\begin{table}[!htb]
\caption{\centering Дисковое пространство, занимаемое тестовыми наборами данных, МБ.}
\small
\centering\begin{tabular}{||c||c|c||}
\hline\hline
Набор данных & Исходные файлы & Данные кластера \\
\hline\hline
Elliptic++ Transactions & 670 & 744 \\
\hline
MOOC User Actions & 48 & 74 \\
\hline
California Road Network & 84 & 122 \\
\hline
Stablecoin ERC20 Transactions & 823 & 901 \\
\hline\hline
\end{tabular}
\label{table:datasetsMemory}
\end{table}

\subsection{Выполнение запросов}

Для получения следующих показателей все запросы выполнялись в одинаковых условиях и по 5 раз; среднее значение
результатов выполнения считается итоговым значением. Характеристики устройства, на котором производились замеры:
\begin{itemize}
  \item объём оперативной памяти: 16 ГБ;
  \item процессор: Intel Core i5 10300H, 2500 МГц;
  \item видеокарта: Nvidia GeForce GTX 1650 Ti.
\end{itemize}

\subsubsection{Время выполнения}

В таблице \ref{table:queryTime} приводится общее время выполнения запросов. В общем времени учитывается синтаксический
разбора текста запроса, непосредственно выполнение и сериализация результатов выполнения.

\begin{table}[!htb]
\caption{\centering Время выполнения запросов, мс.}
\small
\centering\begin{tabular}{||c||c|c|c|c|c||}
\hline\hline
Набор данных & Запрос\,\ref{query1} & Запрос\,\ref{query2} & Запрос\,\ref{query3} & Запрос\,\ref{query4} & Запрос\,\ref{query5} \\
\hline\hline
Elliptic++ Transactions & 27 & 85 & 448 & 3\,072 & 23 \\
\hline
MOOC User Actions & 8\,030 & 202 & 1\,235 & 2989 & 2\,667 \\
\hline
California Road Network & 8\,849 & 1 & 1 & 13\,661 & 30\,540 \\
\hline
Stablecoin ERC20 Transactions & 4\,199 & 1\,093 & 1\,099 & 56\,025 & 5\,546 \\
\hline\hline
\end{tabular}
\label{table:queryTime}
\end{table}

Подчеркнём, что показатели выполнения одного типа запросов на разных наборах тестовых данных не сравнимы между собой.
Указанные наборы разнородные, и запросы раздела \ref{forming} написаны с учётом индивидуальных особенностей каждого
набора так, чтобы отражать наиболее показательные результаты для данного набора, не беря во внимание показатели для
других наборов. Так, например, выбор всех вершин с заданным значением поля (запрос \ref{query2}) на наборе California
Road Network занимает 1 мс, поскольку выбирается единственная вершина по значению предиката \texttt{id} (набор не
обладает нетривиальными предикатами), и в то же время выбор всех вершин набора Stablecoin ERC20 Transactions по значению
предиката \texttt{value} занимает более показательные 1\,093 мс.

Отметим некоторые выделяющиеся показатели таблицы \ref{table:queryTime}. Видно, что в среднем наиболее долго выполняются
запросы типа \ref{query4} --- это ожидаемо, исходя из их определения. Самое продолжительное время выполнения такого
запроса составляет более 56 секунд и соответствует набору данных ERC20 Stablecoin Transactions. Поиск кратчайшего пути
(запрос \ref{query5}) и рекурсивный запрос (\ref{query1}) наиболее продолжительны на наборе данных California Road Network: более 30 и 8 секунд соответственно. Хотя вершины этого набора не обладают атрибутами, его структура достаточно
сложна, чтобы показывать интересные результаты при обходе графа.

\subsubsection{Оперативная память}

В таблице \ref{table:queryMemory} приводятся данные о потреблении оперативной памяти во время выполнения запросов.
Показатели получены как разность свободной оперативной памяти до выполнения запроса и её минимального значения во время
выполнения.

\begin{table}[htb]
\caption{\centering Потребление оперативной памяти при выполнении запросов, МБ.}
\small
\centering\begin{tabular}{||c||c|c|c|c|c||}
\hline\hline
Набор данных & Запрос\,\ref{query1} & Запрос\,\ref{query2} & Запрос\,\ref{query3} & Запрос\,\ref{query4} & Запрос\,\ref{query5} \\
\hline\hline
Elliptic++ Transactions & 20 & 60 & 684 & 1\,941 & 19 \\
\hline
MOOC User Actions & 4\,340 & 112 & 1\,655 & 853 & 1\,839 \\
\hline
California Road Network & 419 & 4 & 1 & 9\,266 & 416 \\
\hline
Stablecoin ERC20 Transactions & 2\,447 & 351 & 954 & 14\,293 & 2\,170 \\
\hline\hline
\end{tabular}
\label{table:queryMemory}
\end{table}

По умолчанию кластер Dgraph может использовать всю свободную оперативную память во время выполнения запроса. Это
особенно заметно при выполнении запроса \ref{query4} на наборе данных Stablecoin ERC20 Transactions: объём потребляемой
оперативной памяти составляет около 14 ГБ и не больше лишь по той причине, что объём всей оперативной памяти основого
устройства равен 16 ГБ.

Видно, что данные таблицы \ref{table:queryMemory} коррелируют с данными таблицы \ref{table:queryTime} в среднем, но не
всегда. Например, с приблизительно одинаковым временем выполнения запроса \ref{query1} на наборах данных MOOC User
Actions и California Road Network (около 8 секунд), объёмы затраченной оперативной памяти различаются практически в 10
раз: 4\,340 МБ и 419 МБ. Это связано со сложной логикой обхода графа MOOC User Actions, где в рекурсивном запросе
рассматриваются 4 предиката (\texttt{performs}, \texttt{\textasciitilde performs}, \texttt{on} и
\texttt{\textasciitilde on}), в то время как при обходе графа California Road Network рассматривается единственный
предикат \texttt{successors}.
%%%%%%%%%%%%%%%%%%%%%%%%%%%%%%%%%%%%%%%%%%%%%%%%%%%%%%%%%%%%%%%%%%%%%%%%%%%%%%%%%%%%%%%%%%%%%%%%%%%%%%%%%%%%%%%%%%%%%%%%
