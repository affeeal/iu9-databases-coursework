% Раздел тестирования, во первых, должен содержать разработку и выполнение тестов, подтверждающих
% работоспособность созданного программного обеспечения. Во-вторых, в нем должны быть приведены
% результаты теоретического или экспериментального исследования, выполненного в ходе курсового
% проектирования. Например, это могут быть результаты, полученные при исследовании математического
% метода, положенного в основу разработанного алгоритма, или оценка временных и иных характеристик
% комплекса программ (алгоритма) в зависимости от особенностей входных данных. Объем раздела
% тестирования составляет 15-20%.

%%%%%%%%%%%%%%%%%%%%%%%%%%%%%%%%%%%%%%%%%%%% TEXT WIDTH %%%%%%%%%%%%%%%%%%%%%%%%%%%%%%%%%%%%%%%%%%%%
\section{Оценка производительности}

\subsection{Статистика датасетов}

Прежде чем представить результаты выполнения запросов, приведём сводную статистику по датасетам.

\subsubsection{Вершины и рёбра}

В таблице \ref{table:datasetsTopology} содержится итоговая информация о вершинах и рёберах датасетов.
Для датасета Elliptic++ Transactions эти данные остались прежними: число вершин есть число транзакций,
число рёбер --- число потоков транзакций. Для датасета MOOC User Actions число вершин складывается из
числа пользователей, действий и онлайн-курсов, а число рёбер равняется числу действий, домноженному на 2,
поскольку каждое действие соединяет одного пользователя и один курс. Для датасета California Road Network
данные также не изменились: число вершин --- число узлов, число рёбер --- число связей. Для датасета
Stablecoin ERC20 Transactions число вершин складывается из числа адресов, полученного вспомогательным
запросом, и числа передач; число рёбер есть трижды число передач, поскольку одна передача связывает
адрес отправителя, адрес получателя и адрес контракта.

\begin{table}[htb]
\caption{\centering Вершины и рёбра датасетов.}
\small
\centering\begin{tabular}{||c||c|c||}
\hline\hline
Датасет & Вершины & Рёбра \\
\hline\hline
Elliptic++ Transactions & 203\,769 & 234\,355 \\
\hline
MOOC User Actions & 418\,893 & 823\,498 \\
\hline
California Road Network & 1\,965\,206 & 2\,766\,607 \\
\hline
Stablecoin ERC20 Transactions & 6\,803\,466 & 15\,840\,393 \\ % 5 280 132 + 1 523 334
\hline\hline
\end{tabular}
\label{table:datasetsTopology}
\end{table}

\subsubsection{Дисковое пространство}

В таблице \ref{table:datasetsMemory} содержатся данные (в мегабайтах) о занимаемом дисковом
пространстве исходными файлами датасетов и данными кластера Dgraph после импорта датасетов.
Данные для исходных файлов тривиально получены суммированием размера файлов. Данные кластера
сняты с директории \texttt{dgraph/p} Docker-контейнера, который разделяют Alpha- и Zero-узлы
кластера.

\begin{table}[htb]
\caption{\centering Занимаемое датасетами дисковое пространство, МБ.}
\small
\centering\begin{tabular}{||c||c|c||}
\hline\hline
Датасет & Исходные файлы & Данные кластера \\
\hline\hline
Elliptic++ Transactions & 670 & 744 \\
\hline
MOOC User Actions & 48 & 74 \\
\hline
California Road Network & 84 & 122 \\
\hline
Stablecoin ERC20 Transactions & 823 & 901 \\
\hline\hline
\end{tabular}
\label{table:datasetsMemory}
\end{table}

\subsection{Выполнение запросов}

Для получения показателей следующих разделов все запросы выполнялись в одинаковых условиях и по 5 раз;
среднее значение результатов выполнения считается итоговым значением.

Характеристики устройства, на котором производилось оценивание:
\begin{itemize}
    \item объём оперативной памяти: 16 ГБ;
    \item процессор: Intel Core i5 10300H, 2500 МГц;
    \item видеокарта: Nvidia GeForce GTX 1650 Ti.
\end{itemize}

\subsubsection{Время выполнения}

В таблице \ref{table:queryTime} содержатся данные о времени выполнения запросов (в миллисекундах).
Под временем понимается общее время выполнения запроса, включающее синтаксический разбор текста
запроса, непосредственно выполнение и сериализацию результатов выполнения.

\begin{table}[htb]
\caption{\centering Время выполнения запросов, мс.}
\small
\centering\begin{tabular}{||c||c|c|c|c|c||}
\hline\hline
\backslashbox{Датасет}{Запрос} & \ref{query1} & \ref{query2} & \ref{query3} & \ref{query4} & \ref{query5} \\
\hline\hline
Elliptic++ Transactions & 27 & 85 & 448 & 3\,072 & 23 \\
\hline
MOOC User Actions & 8\,030 & 202 & 1\,235 & 2989 & 2\,667 \\
\hline
California Road Network & 8\,849 & 1 & 1 & 13\,661 & 30\,540 \\
\hline
Stablecoin ERC20 Transactions & 4\,199 & 1\,093 & 1\,099 & 56\,025 & 5\,546 \\
\hline\hline
\end{tabular}
\label{table:queryTime}
\end{table}

\subsubsection{Оперативная память}

В таблице \ref{table:queryMemory} содержатся данные о потреблении оперативной памяти
(в мегабайтах) во время выполнения запросов. Показатели получены, как разность свободной
оперативной памяти до выполнения запроса и её минимального значения во время выполнения.

\begin{table}[htb]
\caption{\centering Потребление ОЗУ при выполнении запросов, МБ.}
\small
\centering\begin{tabular}{||c||c|c|c|c|c||}
\hline\hline
\backslashbox{Датасет}{Запрос} & \ref{query1} & \ref{query2} & \ref{query3} & \ref{query4} & \ref{query5} \\
\hline\hline
Elliptic++ Transactions & 20 & 60 & 684 & 1\,941 & 19 \\
\hline
MOOC User Actions & 4\,340 & 112 & 1\,655 & 853 & 1\,839 \\
\hline
California Road Network & 419 & 4 & 1 & 9\,266 & 416 \\
\hline
Stablecoin ERC20 Transactions & 2\,447 & 351 & 954 & 14\,293 & 2\,170 \\
\hline\hline
\end{tabular}
\label{table:queryMemory}
\end{table}

%%%%%%%%%%%%%%%%%%%%%%%%%%%%%%%%%%%%%%%%%%%% TEXT WIDTH %%%%%%%%%%%%%%%%%%%%%%%%%%%%%%%%%%%%%%%%%%%%
