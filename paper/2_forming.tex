%%%%%%%%%%%%%%%%%%%%%%%%%%%%%%%%%%%%%%%%%%%%%%%%%%%%%% TEXT WIDTH %%%%%%%%%%%%%%%%%%%%%%%%%%%%%%%%%%%%%%%%%%%%%%%%%%%%%%

\section{Формирование запросов и тестовых данных}

Для оценки производительности СУБД Dgraph над тестовыми данными выполняются следующие типы графовых
запросов\footnote{Запросы обозначаются строчными буквами латинского алфавита для удобной ссылки на них в дальнейшем.}:
\begin{enumerate}[label=(\alph*)]
    \item рекурсивный запрос с фильтрацией по вершинам на пути; \label{query1}
    \item выбор всех вершин с заданным значением поля; \label{query2}
    \item выбор всех вершин с заданным значением поля с фильтрацией по связям и степени вершины; \label{query3}
    \item подсчёт для вершин агрегации некоторого параметра по соседним вершинам с ограничением на значение параметра;
\label{query4}
    \item поиск кратчайшего пути между вершинами с фильтрацией по вершинам на пути. \label{query5}
\end{enumerate}

Графовые запросы выполняются над четырьмя разнородными наборами данных в формате CSV. Далее излагаются описание наборов,
их структура, графовая интерпретация, а также конкретные схема и запросы DQL.

\subsection{Набор данных Elliptic++ Transactions}

\subsubsection{Описание и структура}

Набор Elliptic++ Transactions~\cite{elliptic} содержит данные о 203\,769 транзакциях Bitcoin, их легальности и
потоке. Набор состоит из следующих файлов:
\begin{itemize}
  \item файл \texttt{txs\_features.csv} (663 МБ), в котором определяются основные атрибуты транзакций: временн\'{a}я
метка в целочисленном диапазоне [0, 49], 93 локальных атрибута, описывающих собственную информацию транзакции,
72 агрегированных атрибута, составленных на основе входящих и исходящих транзакций, и 17 дополнительных атрибутов,
отражающих статистическую информацию;
    \item файл \texttt{txs\_classes.csv} (2.3 МБ), в котором определяется класс транзакции. Множество всех транзакций
разбивается на три класса: легальные (42\,019), нелегальные (4\,545) и неопределённые (157\,205);
    \item файл \texttt{txs\_edgelist.csv} (4.3 МБ), в котором хранятся 234\,355 упорядоченных пар идентификаторов
транзакций, отражающих их поток.
\end{itemize}

\subsubsection{Интерпретация и схема}

Исходные данные естественным образом представляются ориентированным графом, вершины которого --- транзакции, а
направленные рёбра соответствуют потоку транзакций. В листинге \ref{lst:elliptic:schema} приводится фрагмент DQL-схемы
набора, где определены предикаты, использующиеся в дальнейшем при выполнении запросов. Исходные идентификаторы
транзакций представляются предикатом \texttt{id}. Все исходящие транзакции из данной транзакции представляются
предикатом \texttt{successors}.
\begin{listing}[!htb]
\caption{Фрагмент DQL-схемы набора данных Elliptic++ Transactions.}
\inputminted[frame=single,fontsize=\footnotesize,linenos,breaklines,xleftmargin=1.5em, breaksymbol=""]{text}
{lst/elliptic/schema}
\label{lst:elliptic:schema}
\end{listing}

\subsubsection{Запросы}

Выбор вершин графа и предикатов, участвующих в запросах, является важнейшей составляющей оценивания. Во многих
ситуациях показательные результаты достигаются, когда в запросе участвуют вершины с наибольшей полустепенью захода или
исхода --- это нужным образом усложняет выполнение запросов. Для обнаружения таких вершин пишутся вспомогательные
запросы.

\paragraph{Запрос \ref{query1}}

В листинге \ref{lst:elliptic:most_successors_transaction} приводится вспомогательный запрос для определения транзакции
с наибольшим числом исходящих транзакций. Предикат \texttt{id} этой транзакции равен 2\,984\,918, а количество
исходящих транзакций --- 472.

\begin{listing}[!htb]
\caption{Запрос для определения транзакции с наибольшим числом исходящих транзакций.}
\inputminted[frame=single,fontsize=\footnotesize,linenos,breaklines,xleftmargin=1.5em, breaksymbol=""]{text}
{lst/elliptic/most_successors_transaction}
\label{lst:elliptic:most_successors_transaction}
\end{listing}

В листинге \ref{lst:elliptic:query1} приводится запрос \ref{query1}. Рекурсивный обход начинается в транзакции с
наибольшим числом исходящих транзакций и завершается на глубине 50. Вершины на пути фильтруются по медианной временной
метке.

\begin{listing}[!htb]
\caption{Запрос \ref{query1} для набора данных Elliptic++ Transactions.}
\inputminted[frame=single,fontsize=\footnotesize,linenos,breaklines,xleftmargin=1.5em, breaksymbol=""]{text}
{lst/elliptic/query1}
\label{lst:elliptic:query1}
\end{listing}

\paragraph{Запрос \ref{query2}}

В листинге \ref{lst:elliptic:query2} приводится запрос \ref{query2}. В запросе выбираются транзакции, легальность
которых не определена --- транзакции со значением 3 предиката \texttt{class}, составляющие 77\% от всех транзакций.

\begin{listing}[!htb]
\caption{Запрос \ref{query2} для набора данных Elliptic++ Transactions.}
\inputminted[frame=single,fontsize=\footnotesize,linenos,breaklines,xleftmargin=1.5em, breaksymbol=""]{text}
{lst/elliptic/query2}
\label{lst:elliptic:query2}
\end{listing}

\paragraph{Запрос \ref{query3}}

В листинге \ref{lst:elliptic:most_predecessors_transaction} приводится вспомогательный запрос для определения
транзакции с наибольшим числом входящих транзакций. Предикат \texttt{id} этой транзакции равен 43\,388\,675, а
количество входящих транзакций --- 284.

\begin{listing}[!htb]
\caption{Запрос для определения транзакции с наибольшим числом входящих транзакций.}
\inputminted[frame=single,fontsize=\footnotesize,linenos,breaklines,xleftmargin=1.5em, breaksymbol=""]{text}
{lst/elliptic/most_predecessors_transaction}
\label{lst:elliptic:most_predecessors_transaction}
\end{listing}

В листинге \ref{lst:elliptic:query3} приводится запрос \ref{query3}. Из транзакций запроса \ref{query2} отбираются те,
что входят в транзакцию, обладающую наибольшим числом входящих транзакций. Полученные транзакции фильтруются по степени 10.

\begin{listing}[!htb]
\caption{Запрос \ref{query3} для набора данных Elliptic++ Transactions.}
\inputminted[frame=single,fontsize=\footnotesize,linenos,breaklines,xleftmargin=1.5em, breaksymbol=""]{text}
{lst/elliptic/query3}
\label{lst:elliptic:query3}
\end{listing}

\paragraph{Запрос \ref{query4}}

В листинге \ref{lst:elliptic:query4} приводится запрос \ref{query4}. Для каждой транзакции подсчитывается среднее
значение поля \texttt{Local\_feature\_1} связанных транзакций, которые удовлетворяют фильтру на \texttt{Local\_feature\_1}.

\begin{listing}[!htb]
\caption{Запрос \ref{query4} для набора данных Elliptic++ Transactions.}
\inputminted[frame=single,fontsize=\footnotesize,linenos,breaklines,xleftmargin=1.5em, breaksymbol=""]{text}
{lst/elliptic/query4}
\label{lst:elliptic:query4}
\end{listing}

\paragraph{Запрос \ref{query5}}

В листинге \ref{lst:elliptic:query5} приводится запрос \ref{query5}. Выполняется поик кратчайшего пути между
транзакциями с наибольшим числом исходящих и входящих транзакций. Транзакции на пути фильтруются по временной
метке.

\begin{listing}[!htb]
\caption{Запрос \ref{query5} для набора данных Elliptic++ Transactions.}
\inputminted[frame=single,fontsize=\footnotesize,linenos,breaklines,xleftmargin=1.5em, breaksymbol=""]{text}
{lst/elliptic/query5}
\label{lst:elliptic:query5}
\end{listing}

\subsection{Набор данных MOOC User Actions}

\subsubsection{Описание и структура}

Набор MOOC User Actions~\cite{mooc} содержит данные о 411\,749 действиях 7\,047 пользователей на платформе
онлайн-курсов MOOC. Каждое действие ассоциировано с одним из 97 онлайн-курсов.

Набор состоит из следующих файлов:
\begin{itemize}
    \item файл \texttt{mooc\_actions.tsv} (11 МБ), в котором пользователи и курсы связываются отношением многие ко многим
посредством действий, и действиям назначаются временные метки;
    \item файл \texttt{mooc\_action\_features.tsv} (35 МБ), в котором определяются 4 вещественнозначных атрибута
действия;
    \item файл \texttt{mooc\_action\_labels.tsv} (3.5 МБ), в котором действиям назначаются бинарные метки соответственно
тому, является ли действие последним действием пользователя на курсе.
\end{itemize}

\subsubsection{Интерпретация и схема}

Исходные данные представляются ориентированным графом, вершины которого --- пользователи, действия и курсы, а рёбра
связывают пользователей с действиями и действия с курсами. В листинге \ref{lst:mooc:schema} приводится соответствующая
DQL-схема. Исходные идентификаторы пользователей, действий и курсов представляются предикатами \texttt{userId}, \texttt{actionId} и
\texttt{targetId} соответственно. Все действия одного пользователя выражаются предикатом \texttt{performs}, а связь
действия и курса описывается предикатом \texttt{on}.

\begin{listing}[!htb]
\caption{DQL-схема набора данных MOOC User Actions.}
\inputminted[frame=single,fontsize=\footnotesize,linenos,breaklines,xleftmargin=1.5em, breaksymbol=""]{text}
{lst/mooc/schema}
\label{lst:mooc:schema}
\end{listing}

\subsubsection{Запросы}

\paragraph{Запрос \ref{query1}}

В листинге \ref{lst:mooc:query1_aux} приводится вспомогательный запрос для определения двух пользователей, совершивших
наибольшее число действий. Ими являются пользователи с \texttt{userId} 1181 (505 действий) и \texttt{userId} 1686
(470 действий).

\begin{listing}[!htb]
\caption{Запрос для определения двух пользователей, совершивших наибольшее число действий.}
\inputminted[frame=single,fontsize=\footnotesize,linenos,breaklines,xleftmargin=1.5em, breaksymbol=""]{text}
{lst/mooc/query1_aux}
\label{lst:mooc:query1_aux}
\end{listing}

В листинге \ref{lst:mooc:query1} приводится запрос \ref{query1}. Рекурсивный обход начинается с пользователя,
совершившего наибольшее число действий, и завершается на глубине 5. В обходе участвуют предикаты \texttt{performs},
\texttt{on}, а также их обратные версии --- предикаты \texttt{\textasciitilde performs} и \texttt{\textasciitilde on}.
Это позволяет проводить обход графа без учёта направления рёбер. Фильтрация действий на пути проводится по временной метке.

\begin{listing}[!htb]
\caption{Запрос \ref{query1} для набора данных MOOC User Actions.}
\inputminted[frame=single,fontsize=\footnotesize,linenos,breaklines,xleftmargin=1.5em, breaksymbol=""]{text}
{lst/mooc/query1}
\label{lst:mooc:query1}
\end{listing}

\paragraph{Запрос \ref{query2}}

В листинге \ref{lst:mooc:query2} приводится запрос \ref{query2}. В выборке участвуют действия, не являющиеся последними
действиями пользователя на курсе. Значение предиката \texttt{class} этих действий равно 0, а их количество составляет 95\%
от количества всех действий.

\begin{listing}[!htb]
\caption{Запрос \ref{query2} для набора данных MOOC User Actions.}
\inputminted[frame=single,fontsize=\footnotesize,linenos,breaklines,xleftmargin=1.5em, breaksymbol=""]{text}
{lst/mooc/query2}
\label{lst:mooc:query2}
\end{listing}

\paragraph{Запрос \ref{query3}}

В листинге \ref{lst:mooc:query3_aux} приводится вспомогательный запрос для определения курса, с которым ассоциировано
наибольшее число действий --- курса с \texttt{targetId} 8 и 19\,474 действиями.

\begin{listing}[!htb]
\caption{Запрос для определения курса, с которым ассоциировано наибольшее число действий.}
\inputminted[frame=single,fontsize=\footnotesize,linenos,breaklines,xleftmargin=1.5em, breaksymbol=""]{text}
{lst/mooc/query3_aux}
\label{lst:mooc:query3_aux}
\end{listing}

В листинге \ref{lst:mooc:query3} приводится запрос \ref{query3}. Из множества действий запроса \ref{query2} выбираются
действия, относящиеся к курсу с \texttt{targetId} 8. Все вершины полученного множества имеют степень 2
(тривиально, по своей структуре); соответствующая проверка добавлена для поддержания единообразия запросов.

\begin{listing}[!htb]
\caption{Запрос \ref{query3} для набора данных MOOC User Actions.}
\inputminted[frame=single,fontsize=\footnotesize,linenos,breaklines,xleftmargin=1.5em, breaksymbol=""]{text}
{lst/mooc/query3}
\label{lst:mooc:query3}
\end{listing}

\paragraph{Запрос \ref{query4}}

В листинге \ref{lst:mooc:query4} приводится запрос \ref{query4}. Для каждого пользователя и курса подсчитывается
среднее значение поля \texttt{feature1} действий, с которыми этот пользователь или курс связан. В подсчёте учитываются
только неотрицательные значения поля \texttt{feature1}.

\begin{listing}[!htb]
\caption{Запрос \ref{query4} для набора данных MOOC User Actions.}
\inputminted[frame=single,fontsize=\footnotesize,linenos,breaklines,xleftmargin=1.5em, breaksymbol=""]{text}
{lst/mooc/query4}
\label{lst:mooc:query4}
\end{listing}

\paragraph{Запрос \ref{query5}}

В листинге \ref{lst:mooc:query5} приводится запрос \ref{query5}. Кратчайший путь между пользователями с
\texttt{userId} 1181 и \texttt{userId} 1686 вычисляется без учёта направления рёбер и с фильтрацией вершин на
пути по временной метке.

\begin{listing}[!htb]
\caption{Запрос \ref{query5} для набора данных MOOC User Actions.}
\inputminted[frame=single,fontsize=\footnotesize,linenos,breaklines,xleftmargin=1.5em, breaksymbol=""]{text}
{lst/mooc/query5}
\label{lst:mooc:query5}
\end{listing}

\subsection{Набор данных California Road Network}

\subsubsection{Описание и структура}

Набор California Road Network~\cite{roadnet} содержит данные о дорожной сети Калифорнии. Набор содержит единственный
исходный файл \texttt{roadNet-CA.txt} (84 МБ), в котором определяются узлы дорожной сети --- пункты назначения или
пересечения дорог, и двусторонние связи между ними. Количество узлов: 1\,965\,206, и они не имеют атрибутов;
количество связей: 2\,766\,607.

\subsubsection{Интерпретация и схема}

Исходные данные представляются обыкновенным неориентированным графом. DQL-схема набор приводится в
листинге \ref{lst:roadnet:schema}. Предикат \texttt{id} представляет исходный идентификатор узла, а предикат
\texttt{successors} --- связанные узлы.

\begin{listing}[!htb]
\caption{DQL-схема набора данных California Road Network.}
\inputminted[frame=single,fontsize=\footnotesize,linenos,breaklines,xleftmargin=1.5em, breaksymbol=""]{text}
{lst/roadnet/schema}
\label{lst:roadnet:schema}
\end{listing}

\subsubsection{Запросы}

\paragraph{Запрос \ref{query1}}

В листинге \ref{lst:roadnet:query1_aux} приводится вспомогательный запрос для определения двух узлов с наибольшей
степенью. Такими являются узлы с \texttt{id} 562\,818 (степень 12) и \texttt{id} 521\,168 (степень 10).

\begin{listing}[!htb]
\caption{Вспомогательный запрос для определения двух узлов наибольшей степени.}
\inputminted[frame=single,fontsize=\footnotesize,linenos,breaklines,xleftmargin=1.5em, breaksymbol=""]{text}
{lst/roadnet/query1_aux}
\label{lst:roadnet:query1_aux}
\end{listing}

В листинге \ref{lst:roadnet:query1} приводится запрос \ref{query1}. Поскольку узлы не обладают нетривиальными предикатами,
для имитации этих предикатов используется \texttt{id}. Так, рекурсивный запрос начинается в узле с \texttt{id} 562\,818,
завершается на глубине 50, и вершины на пути фильтруются по значению предиката \texttt{id}.

\begin{listing}[!htb]
\caption{Запрос \ref{query1} для набора данных California Road Network.}
\inputminted[frame=single,fontsize=\footnotesize,linenos,breaklines,xleftmargin=1.5em, breaksymbol=""]{text}
{lst/roadnet/query1}
\label{lst:roadnet:query1}
\end{listing}

\paragraph{Запрос \ref{query2}}

В листинге \ref{lst:roadnet:query2} приводится запрос \ref{query2}. Выборка запроса тривиальная и совершается по
значению 562\,818 предиката \texttt{id}.

\begin{listing}[!htb]
\caption{Запрос \ref{query2} для набора данных California Road Network.}
\inputminted[frame=single,fontsize=\footnotesize,linenos,breaklines,xleftmargin=1.5em, breaksymbol=""]{text}
{lst/roadnet/query2}
\label{lst:roadnet:query2}
\end{listing}

\paragraph{Запрос \ref{query3}}

В листинге \ref{lst:roadnet:query3} приводится запрос \ref{query3}. Для узла с \texttt{id} 562\,818 проверяется наличие
связи с узлом, \texttt{id} которого равен 562\,826 (в действительности это непосредственный сосед исходного узла). Далее
проверяется, что степень узла совпадает с ожидаемой --- с 12.

\begin{listing}[!htb]
\caption{Запрос \ref{query3} для набора данных California Road Network.}
\inputminted[frame=single,fontsize=\footnotesize,linenos,breaklines,xleftmargin=1.5em, breaksymbol=""]{text}
{lst/roadnet/query3}
\label{lst:roadnet:query3}
\end{listing}

\paragraph{Запрос \ref{query4}}

В листинге \ref{lst:roadnet:query4} приводится запрос \ref{query4}. Для каждого узла вычисляется среднее арифметическое
значений \texttt{id} соседних узлов, удовлетворяющих фильтру.

\begin{listing}[!htb]
\caption{Запрос \ref{query4} для набора данных California Road Network.}
\inputminted[frame=single,fontsize=\footnotesize,linenos,breaklines,xleftmargin=1.5em, breaksymbol=""]{text}
{lst/roadnet/query4}
\label{lst:roadnet:query4}
\end{listing}

\paragraph{Запрос \ref{query5}}

В листинге \ref{lst:roadnet:query5} приводится запрос \ref{query5}. Поиск кратчайшего пути выполняется между узлами с
наибольшей степенью. Фильтрация производится по предикату \texttt{id}.

\begin{listing}[!htb]
\caption{Запрос \ref{query5} для набора данных California Road Network.}
\inputminted[frame=single,fontsize=\footnotesize,linenos,breaklines,xleftmargin=1.5em, breaksymbol=""]{text}
{lst/roadnet/query5}
\label{lst:roadnet:query5}
\end{listing}

\subsection{Набор данных Stablecoin ERC20 Transactions} \label{datasetERC20}

\subsubsection{Описание и структура}

\textit{Стейблкоинами} называют специальные токены, предназначенные для поддержания фиксированной стоимости в течение
долгого времени. Набор Stablecoin ERC20 Transactions~\cite{erc20} содержит данные о более чем 70 миллионах транзакций ERC20
пяти популярных стейблкоинов. Описания транзакций распределены по трём файлам набора, имеющим одинаковую структуру:
\texttt{token\_transfers.csv} (823 МБ), \texttt{token\_transfers\_V2.0.0.csv} (4.4 ГБ)
и \texttt{token\_transfers\_V3.0.0.csv} (5.6 ГБ). В файлах определяются \textit{переводы} (transfers), совершаемые в
рамках транзакций. Описание каждого перевода включает номер блока транзакции, индекс транзакции, количество
переданных стейблкоинов, временную метку, а также адреса отправителя, получателя и контракта, определяющего стейблкоин.
Набор также содержит файл с описанием событий, повлиявших на работу сети, файлы с изменением стоимости стейблкоинов,
однако объём информации в них незначительный (суммарно, 60 КБ), и этими данными можно пренебречь. Далее работа ведётся
только с файлом \texttt{token\_transfers.csv}.

\subsubsection{Интерпретация и схема}

Исходные данные можно представить ориентированным графом. В отличие от набора данных Elliptic++ Transactions, где транзакции
являются единственными вершинами в графе и непосредственно связаны друг с другом, здесь вершины графа ---
переводы и адреса отправителей, получателей или контрактов. Таким образом, переводы являются посредниками при
выражении отношения многие ко многим между адресами отправителей и получателей. Направленные рёбра в графе связывают
\begin{itemize}
    \item адрес отправителя и перевод;
    \item перевод и адрес получателя;
    \item перевод и адрес контракта.
\end{itemize}
В листинге \ref{lst:erc20:schema} приводится DQL-схема набора. Исходные адреса отправителей, получателей и контрактов
представляются предикатом \texttt{address}. Все переводы некоторого отправителя выражаются предикатом \texttt{from}. Связь
перевода и получателя описывается предикатом \texttt{to}, и связь перевода и контракта --- предикатом \texttt{contract}.

\begin{listing}[!htb]
\caption{DQL-схема набора данных Stablecoin ERC20 Transactions.}
\inputminted[frame=single,fontsize=\footnotesize,linenos,breaklines,xleftmargin=1.5em, breaksymbol=""]{text}
{lst/erc20/schema}
\label{lst:erc20:schema}
\end{listing}

\subsubsection{Запросы}

\paragraph{Запрос \ref{query1}}

В листинге \ref{lst:erc20:greatest_transfers} приводится вспомогательный запрос для определения адреса отправителя
наибольшего числа переводов. Соответствующий предикат \texttt{address} равен
\texttt{0x74de5d4fcbf63e00296fd95d33236b9794016631}, а количество переводов --- 147\,437.

\begin{listing}[!htb]
\caption{Вспомогательный запрос для определения адреса отправителя наибольшего числа переводов.}
\inputminted[frame=single,fontsize=\footnotesize,linenos,breaklines,xleftmargin=1.5em, breaksymbol=""]{text}
{lst/erc20/greatest_transfers}
\label{lst:erc20:greatest_transfers}
\end{listing}

В листинге \ref{lst:erc20:query1} приводится запрос \ref{query1}. Рекурсивный обход выполняется от адреса отправителя
наибольшего числа переводов и до глубины 3. На пути рассматриваются предикат \texttt{from} с фильтрацией по временной
метке и предикат \texttt{to}.

\begin{listing}[!htb]
\caption{Запрос \ref{query1} для набора данных Stablecoins ERC20 Transactions.}
\inputminted[frame=single,fontsize=\footnotesize,linenos,breaklines,xleftmargin=1.5em, breaksymbol=""]{text}
{lst/erc20/query1}
\label{lst:erc20:query1}
\end{listing}

\paragraph{Запрос \ref{query2}}

В листинге \ref{lst:erc20:query2} приводится запрос \ref{query2}. Переводы отбираются по числу стейблкоинов.

\begin{listing}[!htb]
\caption{Запрос \ref{query2} для набора данных Stablecoins ERC20 Transactions.}
\inputminted[frame=single,fontsize=\footnotesize,linenos,breaklines,xleftmargin=1.5em, breaksymbol=""]{text}
{lst/erc20/query2}
\label{lst:erc20:query2}
\end{listing}

\paragraph{Запрос \ref{query3}}

В листинге \ref{lst:erc20:most_transfers_contract} приводится вспомогательный запрос для определения адреса контракта
с наибольшим числом переводов. Соответствующий \texttt{address} равен \texttt{0xdac17f958d2ee523a2206206994597c13d831ec7}, а
количество переводов --- 2\,652\,911.

\begin{listing}[!htb]
\caption{Вспомогательный запрос для определения адреса контракта с наибольшим числом переводов.}
\inputminted[frame=single,fontsize=\footnotesize,linenos,breaklines,xleftmargin=1.5em, breaksymbol=""]{text}
{lst/erc20/most_transfers_contract}
\label{lst:erc20:most_transfers_contract}
\end{listing}

В листинге \ref{lst:erc20:query3} приводится запрос \ref{query3}. Переводы запроса \ref{query2} фильтруются по
адресу контракта с наибольшим числом переводов. Полученные вершины фильтруются по степени 2.

\begin{listing}[!htb]
\caption{Запрос \ref{query3} для набора данных Stablecoins ERC20 Transactions.}
\inputminted[frame=single,fontsize=\footnotesize,linenos,breaklines,xleftmargin=1.5em, breaksymbol=""]{text}
{lst/erc20/query3}
\label{lst:erc20:query3}
\end{listing}

\paragraph{Запрос \ref{query4}}

В листинге \ref{lst:erc20:query4} приводится запрос \ref{query4}. Для каждого адреса подсчитывается среднее значение
стейблкоинов связанных переводов, которые удовлетворяют фильтру.

\begin{listing}[!htb]
\caption{Запрос \ref{query4} для набора данных Stablecoins ERC20 Transactions.}
\inputminted[frame=single,fontsize=\footnotesize,linenos,breaklines,xleftmargin=1.5em, breaksymbol=""]{text}
{lst/erc20/query4}
\label{lst:erc20:query4}
\end{listing}

\paragraph{Запрос \ref{query5}}

В листинге \ref{lst:erc20:query5} приводится запрос \ref{query5}. Выполняется поиск кратчайшего пути между адресом
отправителя наибольшего числа контрактов и адресом 0, регулярно участвующим в переводах. Вершины на пути фильтруются
по временной метке.

% TODO: Вытащить на предыдущую страницу
\begin{listing}[!htb]
\caption{Запрос \ref{query5} для набора данных Stablecoins ERC20 Transactions.}
\inputminted[frame=single,fontsize=\footnotesize,linenos,breaklines,xleftmargin=1.5em, breaksymbol=""]{text}
{lst/erc20/query5}
\label{lst:erc20:query5}
\end{listing}
%%%%%%%%%%%%%%%%%%%%%%%%%%%%%%%%%%%%%%%%%%%%%%%%%%%%%% TEXT WIDTH %%%%%%%%%%%%%%%%%%%%%%%%%%%%%%%%%%%%%%%%%%%%%%%%%%%%%%
